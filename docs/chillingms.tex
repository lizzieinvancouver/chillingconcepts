\documentclass[11pt]{article}
\usepackage[top=1.00in, bottom=1.0in, left=1in, right=1in]{geometry}
\renewcommand{\baselinestretch}{1.1}
\usepackage{graphicx}
\usepackage{natbib}
\usepackage{amsmath}
\usepackage{parskip}

\def\labelitemi{--}
\parindent=0pt

\begin{document}
\bibliographystyle{/Users/Lizzie/Documents/EndnoteRelated/Bibtex/styles/besjournals}
\renewcommand{\refname}{\CHead{}}

Plant leafout each spring has shifted weeks earlier in many regions due to warming from anthropogenic climate change, with consequences for a suite of ecosystem services, including carbon storage CITES.  What exact processes drive this trend, however, has come under increasing debate as increasing research suggests winter warming may slow or stall this advance CITES. Such reports focus on a two-step model of leafout where plants first require cool winter temperatures, often called `chilling,' before the can accumulate enough warm temperatures---`forcing'---to leafout each year. But this model of leafout---especially its chilling component---is only one of many models proposed (at least XX exist, see CITES) since the concept was first introduced. 

How plants in temperature-limited systems avoid leafout during warm spells in the winter has long been debated by plant biologists. Most work to date has focused on the idea that plants enter some form of dormancy in the fall, which is the released---likely through chilling---before spring warm temperatures began. Chilling can be critical to fruit production in many woody plant crops, with research on peaches planted into warmer climates well outside (e.g., Florida, Israel) that lacked cool winters leading to extremely low fruitset because most flower buds did not burst (CITES). This research underlies much of the models of chilling used today through crops and wild tree species, but has rarely (if ever) been well tested, resulting in many models of chilling today. Current models of chilling predict a full suite of possible future leafout; chilling in the same location under different models can easily predict significantly increased or greatly reduced chilling and thus greatly advanced or slowed leafout in the future (CITES GuyChuine). 



How plants in temperature-limited systems enter and exit winter dormancy each year 


2. What is chilling? 

Though early debates considered whether plants were truly dormant or only growing slow (`rest' versus `quiescent,' CITES), today most research assumes this model.

\end{document}
