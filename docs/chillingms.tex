\documentclass[11pt]{article}
\usepackage[top=1.00in, bottom=1.0in, left=1in, right=1in]{geometry}
\renewcommand{\baselinestretch}{1.1}
\usepackage{graphicx}
\usepackage{natbib}
\usepackage{amsmath}
\usepackage{gensymb}
\usepackage{parskip}
\usepackage{xcolor}


\def\labelitemi{--}
\parindent=0pt

\begin{document}
\renewcommand{\refname}{\CHead{}}


\title{The problem and promise of \\ modeling plant leafout under warmer winters} %  How overconfident models of tree leafout have stalled progress... Infinite models of leafout given warming winters...? % I don't love this title.
\author{E.M. Wolkovich$^1$, Justin Ngo$^1$, Victor van der Meersch$^1$, Jonathan Auerbach$^2$} % also possibly: , Ron Guy ... Frederik Baumgarten$^1$ though he has been so MIA, I am not sure if he is interested and this paper is unlike the one we proposed to work on together. 
\date{\today}
\maketitle

$^1$ Forest \& Conservation Sciences, Faculty of Forestry, University of British Columbia, 2424 Main Mall, Vancouver, BC V6T 1Z4, Canada\\
$^2$ Department of Statistics, George Mason University, 4511 Patriot Circle, Fairfax, VA 22030, United States\\

% As of 22 Nov 2025, I have notes on next steps in _dothiChilling.md

\begin{abstract} % 213 words
How warmer winters affect plant leafout has major implications for carbon storage and ecosystem change, but is poorly understood. 
Recent evidence of a slow-down in the advance of spring leafout with increasing anthropogenic climate change has reinvigorated a decades-old debate over winter `chillling'---the concept that woody temperate plants need sufficient winter cool temperatures to budburst each spring. Decades of research has argued that chilling is a critical plant process---one that is never observed but often assumed. 

% Yet current models of chilling make contradictory predictions, 
% JAOct2025: diverse predictions or contradictory predictions/lack of scientific consensus
% with different models given the same data often predicting everything from advanced to slowed leafout. Such variability highlights how little we understand the critical process of `chilling,' which is never observed but often assumed.

% JAOct2025: I think this is great so far. The problem from my perspective is that if this is where the field was before our work, the following sentences do not sound like much of a contribution, unless this is a review paper. Three options: 1. Change the next sentence ``We show" to ``We review". 2. We could replace the previous two sentences ``Yet current models ... but often assumed'' with ``But what if anything does a chilling mechanism reveal about observable plant behavior/phenology?" or 3. We concede that the field has long recognized the non-identifiability problem, but we argue that the failure to make it precise (as we do here) has caused researcher after researcher to reinvoke chilling, which like other theoretical constructs such as the ``ether'', has become a placeholder that does not lead to unique falsifiable predictions and thus stalled scientific progress.
% emwReply: I did (1) and used some of (3), but it should be (2) conceptually, however,  I think we need to set up the problem more concretely than: But what if anything does a chilling mechanism reveal about observable plant behavior/phenology (and it's not 3: since I don't think people know that they have non-identifiable models). 
We review how current work, which routinely models `chilling' with highly limited data has likely led to the proliferation of models---and thus forecasts---that yield little additional insight about observable plant behavior. The result is that chilling, like other theoretical constructs such as the ``ether'', has become a placeholder that does not lead to unique  predictions and thus has stalled scientific progress. We argue that, until we overcome this,  new insights and approaches from molecular biology that could better define the mechanism of chilling will likely lead to only limited progress. By understanding the limits of chilling as a theory, we can both 1. build better models by looking for the necessary insight from molecular biology to fill in the details, and 2. design better experiments to test those theories.
% We conclude that integrating new insights and approaches from molecular biology could better define the mechanism of chilling leading to models that make falsifiable predictions. Exposing the gaps in current mechanistic and statistical models at the same time new experiments yield richer, more informative data could greatly reduce the number of competing models of chilling that we have today---advancing our fundamental understanding of plant dormancy---and lead to much improved forecasts from crops to forest trees.  
\end{abstract}
% ... Such extreme variability of spring phenology predicted from current models suggests fundamental and major gaps in our understanding. Indeed, new research suggests major flaws in these models as currently applied, including multiple papers now suggesting estimates of `chilling' could easily be artifacts of poor models or correlated observational climate data \citep{decsens,gao2024}. 

% Main text: 3800; Box 1: 500.
Plant leafout in the spring has shifted weeks earlier in many regions due to warming from anthropogenic climate change with consequences for a suite of ecosystem services, including carbon sequestration \citep{keenan2014net,ipcc2022}. Which underlying processes drive this trend, however, is debated, as recent research suggests winter warming may slow or stall this advance \citep{fu2015,piao2017}. Such reports focus on a two-step model of leafout where plants first require cool winter temperatures, often called `chilling,' before they can accumulate enough warm temperatures---`forcing'---to leafout each year. This model of chilling is only one of many models proposed since the concept was first introduced \citep[][]{basler2016evaluating,hufkens2018integrated}. But, while forcing ends in an observable event such as leafout, the day plants achieve their chilling requirements coincides with no known measurable event. 

Current models of chilling are consistent with a diverse suite of possible future leafout. Chilling in the same location under different models can easily forecast significantly increased or greatly reduced chilling and thus greatly advanced or slowed leafout \citep{guy2014,chuine2016}. This variability often occurs within any one model of chilling, given different parameter values. Such extreme variability of spring phenology predicted from current models suggests fundamental gaps in our understanding. While deacdes of experimental evidence suggest that winter temperatures impact leafout \citep{charrier2015,baum2021} new research has highlighted major flaws in these models as currently applied, including multiple papers now suggesting estimates of `chilling' could easily be artifacts of poor models or correlated observational climate data \citep{decsens,gao2024}. 

Now appears an opportune time to address the problems in our concept of chilling in models of leafout. Because accurate forecasts of spring phenology are critical for carbon storage, many crops and a number of other services important to humans, there is widespread interest in improving chilling models \citep{Luedeling2015Acta,chuine2016}. At the same time, new results from molecular and cellular studies of dormancy are providing new insights into when and how chilling works \citep{pan2023epigenetic,zhu2021cold}, which could rule in---or out---some of the myriad current models. 
Here, we review the concept of chilling, its origins and potential problems, as well as new opportunities for major advances and how shifting current practices could accelerate progress. We argue that revisiting the chilling mechanism has two benefits: 1. building better models, and 2. designing better experiments. % In particular, we discuss how new insights from molecular and cellular studies combined with modern approaches to building biological models could revolutionize our understanding of chilling. 

\section*{What is chilling?}
% JAOct2025:  Great section, would not cut.

%addbook
How plants in temperature-limited systems avoid leafout during warm spells in the winter has long been debated by plant biologists \citep[e.g.,][]{lamb1948effect,weinberger}. Most work to date has focused on the idea that plants enter some form of dormancy in the fall, which is then released before warm temperatures in the spring begin. This idea hypothesizes that the slow accumulation of cool temperatures---or chilling---over the winter extends through periods of short warm spells in the winter and thus prevents leafout before spring. 

Much of our fundamental understanding of chilling comes from studies on temperate woody fruit crops where chilling can be critical to yield. Peach trees planted into warmer climates well outside their range often have extremely low fruitset because most flower buds do not burst \citep{weinberger,overcash1955effects,erez1971improved}. Initial studies of this phenomenon with related experiments---where cut ends of dormant branches (cuttings) exposed to cooler temperatures in chambers burst more fully and more quickly---underlies most of the models of chilling used today for crops and wild tree species \citep[][]{weinberger,ospreebbms}. %, but these have rarely---if ever---has been well tested. 

The term `chilling' is now used across numerous fields in plant biology to refer to a process where dormant buds exposed to cool temperatures accelerate a phenological event that later occurs after warm temperatures. Focused on how chilling can accelerate events, researchers have calculated `chilling' required for leafout of forest trees from cutting experiments similar to those used for peaches \citep[reviewed in][]{ospreebbms}, ground observations of budburst \citep{Luedeling2009}, and satellite measures of greenup \citep{kaduk2011predicting}. These estimates rely on phenological models that have become critical across a suite of fields, from climatology where modeling how vegetation on the land surface responds to anthropogenic climate change affects carbon storage, to crop biology, where estimated chill units guide growers in which specific cultivar to plant, and has led to the cross-disciplinary field of phenology \citep{Schwartz:1994he,Cleland:2007or,pmp}. % A suite of studies on forest trees have also estimated chilling using cuttings of tree branches during the winter in experiments where `chilling' is manipulated either in controlled environment chambers or through the sequential removal of cuttings CITES. While most of these studies focus on leafout, a similar suite of studies on crops has focused on flower burst. 

Alongside these more macro-scale studies of chilling, molecular approaches have also examined chilling. Many studies have focused on vernalization---cool temperatures required for flowering \citep{kim2009vernalization}---in \emph{Arabidopsis thaliana}, with studies in woody species, especially \emph{Populus} examining chilling before budbreak \citep[][]{azeez2021early,cai2024molecular}. These studies generally use controlled temperatures to vary the hypothesized amount of chilling then examine molecular and cellular responses \citep[e.g.,][]{pan2021aba,azeez2021early,cai2024molecular}.
% Versions of plants with knock-outs of specific genes have helped identify genes that underlie dormancy and other processes related to chilling \citep{songstad2017genome}. 

Today these studies have led to over 30 basic models where accumulated chilling releases plants from dormancy and hundreds more when considering different species and cultivars \citep[][]{basler2016evaluating,hufkens2018integrated}. Though early debates considered whether plants were truly dormant or only growing slow \citep[`dormancy' or `rest' versus `quiescent';][]{considine2016language}, today most research assumes a model with two phases of dormancy: endodormany---where chilling occurs---and a period after but before the observed event, called ecodormancy (Fig. \ref{fig:modelsketch}).  In most models, chilling can only be accumulated under certain temperatures---traditionally above zero but below 10\degree C---with certain temperatures 
being optimal for the most rapid accumulation of `chill units,' where some unknown sum of chill units breaks endo-dormancy. Which temperatures are most effective at providing chilling is a common question addressed in experiments, with different experiments providing different answers \citep{vitasselev,baum2021}. %vvdm25oct2025: maybe you could cut the next sentence related to GDD?
% This mirrors growing degree days in many ways, where a lower temperature (e.g., 5 or 10\degree C base temperature) is too cold for forcing units to accumulate and plants need some total sum of such units to leafout or flower, but has added complexity given the importance of both the lower and upper temperature thresholds for `chill units' \citep[whereas growing degree day models can often ignore the upper threshold, estimated at 25\degree C or above, as it is rarely reached in natural spring conditions,][]{mcmaster1997growing,li2021comparisons}. % This has led to debate recently on whether sub-zero temperatures can provide chilling as one recent study appeared to show \citep{baum2021}, supporting previous research \citep{Jones:2012,Sonsteby:2014aa}, but in contrast to other studies \citep{lamb1948effect,cook2005freezing,Man:2010aa}. These studies, however, vary in a number of additional factors including the species and populations they study. 

Further complexity comes from the hypothesized diversity of these temperature thresholds and sums across species and populations. Most assume different species require different sums of chill units, and may have different lower, upper and optimal chill temperatures. Within species, populations may require different sums of chill units, with populations in more mild climates---where warm interruptions in the winter are more common---requiring more chill units than those in areas with cold winters, where temperatures rarely rise above zero before spring \citep{campbell1979,leinonen1996dependence}. 

\section*{The problem with chilling}
Chilling is a latent, unobserved process. Typically, %little progress towards elucidating this process over the past 40 years \citep{chuine2016,hanninen2019experiments}. 
%Most models of chilling used today for estimating budburst in crops, land surface modeling etc. are process-based, meaning they attempt to represent the underlying mechanistic process. 
%This underling process and the models of it were developed in the 1970s \citep{richardson1974} following several decades of new work---especially on peaches and other temperate tree fruit crops---and much debate \citep{dormtreeproc}. Since then many more models have been proposed \citep{luedeling2012chilling,chuine2016}, but the original models still fit most data well compared to newer models \citep{basler2016evaluating,chuine2016} and are widely used in major studies \citep[e.g.,][]{richardson1974,chuine2016,ospreebbms}. One of the main reasons for this stalled progress is that almost all models have more unobserved components of the mechanistic process than they can fit with current data. 
%Thus, cannot actually estimate the parameters in them, and are fundamentally non-identifiable (see Box: Why has progress on modeling stalled for decades?).
% Experiments and models focused on chilling describe an unobserved process \citep{chuine2016}, with critical differences  between the two approaches. 
%
% Current models use limited data to estimate a hypothesized accumulation that triggers an unobserved event---currently often described as the release of endo-dormancy---which then leads to another accumulation that leads to leafout or flowering. 
`chilling' describes the physiological phase (endo-dormancy) in which a plant experiences environmental temperatures that induce progress towards the next physiological phase (eco-dormancy) that ends in budburst (Fig. \ref{fig:modelsketch}). The problem is that the transition from endo- to eco-dormancy corresponds with no clearly measurable phenomenon, and thus the properties of each period cannot be determined from the observed leafout, even under experimental conditions in which the temperature can be manipulated. The parameters of a model governing this system are said to be underdetermined or unidentified. 

For example, consider a simple model in which one would need to estimate the minimum and maximum temperatures that allow chilling to accumulate (two parameters) and the total sum of those temperature units needed to trigger a shift into the next physiological phase (often called, `endo-dormancy break,' for one additional parameter for three total). Models then need to estimate when plants start and stop accumulating (two more parameters). An experiment that raises temperatures and observes an earlier leafout could be explained by more rapid chilling accumulation leading to earlier endo-dormancy break, more rapid ecodormancy break, or an almost limitless mix of the two. % Even in some of the simplest models, estimates of what temperatures accumulate `chilling' and how much chilling is required to shift physiological stages often occupy multiple divergent options. A common example of this is one outcome where the range of temperatures that allow chilling to accumulate is wide and thus the threshold amount of chilling large, or an outcome where the range is smaller and, thus, the threshold lower. Yet researchers often only present one of these outcomes \citep{chuine2016}, versus showing the full range of possible values---that is, the full uncertainty. 

To address this, models often assign the start date of the endo-dormancy as known (e.g., starting 1 September in the northern hemisphere) and rely on assumptions to set the end date (see Box: Why has progress on modeling stalled for decades?). A common assumption, developed by early work on peaches, is that high and rapid budburst (leaf or flower) is evidence that chilling has been met \citep[i.e. endo-dormancy has ended][]{erez1971}. While this assumption is widely used, it is rarely if ever tested beyond the early work on peaches. This approach of assigning some unknown parameters as known has the benefit of avoiding adding more unknown model parameters, but it also has led researchers to be overly confident in a model of chilling where more is actually unobserved and unknown than acknowledged. 

Hidden assumptions and numerous parameters can easily drive diverging models. Even if we assume high and rapid percent budburst signals sufficient chilling, most models today include parameters that cannot be uniquely identified with current data.  Given experiments and models have suggested many variants on a more complicated model of chilling---for example minimum, maximum and optimal temperatures, or high temperatures that reduce previous accumulation \citep[Fig. \ref{fig:modelsketch}, and see][]{lued2011,luedeling2012chilling,chuine2016}---current data are relatively uninformative to try to estimate all the parameters the models include. Further, recent models have often relied on even less data \citep{hanninen2019experiments}. Many current methods use only observational data of the timing of leafout (or flowering) to attempt to estimate a model of chilling for different species or locations and project it forward to understand effects of anthropogenic climate change \citep{lued2011,luedeling2012chilling,gao2024}. Perhaps not surprisingly then, which model is deemed best varies strongly by method and approach \citep{Caffarra:2011qf,basler2016evaluating,hufkens2018integrated}, with no clear pattern. 


\section*{New molecular insights could reshape the field and its models}
% JAOct2025:  Also great, would not cut.

% Molecular insights have long helped crop and forest tree models of chilling CITES, but recent findings appear especially promising to build a more unified and coherent model. 
New molecular research provides hope that the transition from endo- to eco-dormancy is measurable. Molecular insights have long contributed to crop and forest tree models of chilling \citep{chuinearees}. Decades of work on vernalization have outlined the pathways---and genes---that lead to flowering only after winter's cool temperatures in biennial (herbaceous) populations of \emph{Arabidopsis thaliana} \citep[Fig. \ref{fig:modelsketch},][]{Wilczek:2009oa,kim2009vernalization}. Research has linked some of these pathways to similar ones in woody species, and have also highlighted the sugar callose (1,3-$\beta$-{\sc D}-glucan) as potentially pivotal for chilling \citep{vanderschoot2014,pan2021aba}. Multiple studies across multiple species have now shown that (1) lower temperatures appear to degrade callose and (2) the release/loss of callose appears to re-start cell-to-cell communication before budburst \citep{vanderschoot2014}. Taken together, these results suggest the loss of callose---generally degraded through 1,3-$\beta$-glucanases (a group of enzymes)---may be an indicator of endo-dormancy release, though other factors, such as  ABA, also often change at the same time \citep[][]{tylewicz2018photoperiodic,pan2021aba}, and may provide a similar observable signal of endo-dormancy release \citep{rinne2018,andre2022populus}. 
% New cellular and molecular results could potentially reduce this complexity, and help narrow the set of possible models.  We argue that two new major insights from molecular studies---the importance of callose and temperature-dependent growth---could limit the chilling models considered today, and reshape the experiments tree biologists use to determine `chilling.' 

% MOVE this text to a figure caption maybe?
% The hypothesis that the sugar callose may play a pivotal role in bud endodormancy has been suggested for over a decade \citep{rinne2011}, but recent studies have yielded increasing support for this hypothesis \citep{vanderschoot2014,pan2023epigenetic}. Callose (1,3-$\beta$-{\sc D}-glucan), which is synthesized for various reasons in plants (e.g., at the site of infection to prevent pathogens), also appears to block plasmodesmata in dormant buds. 

If callose is functionally a major controller of endo-dormancy and its release, then chilling models could be limited to those that match the idea of glucanase degrading callose---meaning models that include a temperature range over which the enzyme is active (Fig. \ref{fig:modelsketch}). In contrast, models using simple temperature thresholds (e.g., all hours below -5\degree C equally allow chilling) would appear less biologically accurate, as enzymes generally do not work over such a wide range of temperatures. 

Other new molecular insights similarly suggest that such simplified temperature metrics used in many chilling models may not map to molecular realities. For example, new work on how slow growth itself may act  a `long-term thermosensor' \citep{zhao2020temperature} 
adds to an increasing number of molecular studies that suggest plants integrate long-term thermo-sensing in the winter alongside responses to short-term temperatures \citep{antoniou2021feeling,Satake2022}. The best models of leafout may thus need to integrate across multiple timescales. 

New molecular results could easily add complexity to models of spring phenology that are already challenged by too much complexity. But they could also begin to rule out models by focusing on new experiments and modeling approaches that target the major problems facing models of chilling---if the field is more open about those issues and open to new approaches. 
% especially given how little of the dormancy process we currently observe.  


% Models of chilling and forcing CITEFiG generally assume accumulation processes, mirroring processes that may occur at the cellular level (e.g., accumulation of callose, followed by accumulation glucanases that degrade it). 
% Another major insight from molecular studies suggests that temperature-dependent growth may determine the build-up---or dilution---of cellular components (e.g., proteins) that determine dormancy \citep{zhao2020temperature,antoniou2021feeling}. While many molecular models focus on conditions that increase protein synthesis, new research suggests temperature dependent growth could explain patterns of accumulation and dilution in proteins that trigger vernalization \citep{zhao2020temperature}. In this way, slow growth in the winter may act as a `long-term thermosensor.' If these results hold up across other studies and species, they would suggest that temperature synthesis ranges and optima for different molecules may be less apparent---and potentially less critical to model---when growth regulates concentration \citep{zhao2020temperature}.
% Something about how both callose and temperature dependent growth mean you could go forward and then back (lose callose... gain it back), which adds complexity ... but has been long hypothesized ... but also means models alone without better observed data in experiments will NEVER save us. 


% vvdm25oct: do we need to explain more why having divergent predictions is not enough to falsify the models? Maybe a clear explanation is missing, that would improve the flow throughout the manuscript? In my head it's something like this, but I would like to know if this is what you had in mind: Why the discrepancies across the model predictions are not sufficient to discard some of the models? Because the predictions are not truly falsifiable, since many parameters are arbitrarily fixed or not identifiable. And thus the predictions by the model do not reflect the underlying theory or hypotheses that the modeller aims to represent. Or they only are a bias representation of these hypotheses (e.g.: because of all the unstated hypotheses, or all the hidden modeling choice that the modeller has to take to overcome the non-identifiability issue---such as Isabelle who discards some parameter optimisation outputs because they don't align with her domain expertise). So to reformulate: the non-identifiability (which lead to fixed parameters + subjective choices, similar to parameter tuning?) weakens the link between the model and the theory. The models do not consistently represent the hypotheses we believe they are based on. This prevents a good fit to be a good empirical validation of these hypotheses. I don't know if it makes sense? I know you want to cut words, but I feel this is just a reformulation issue that would not lengthen the manuscrit
% vvdm25oct: after writing this, I read JA feedback and I think it could kind of help answering his comment: "As of right now, this section seems unrelated to the unidentifiability problem [...]"
% chatting with vvdm25Nov2025: why diff predictions not handy? because most models make ALL the predictions (not each model makes truly diff ones)? Why is this? Because of non-identifiability! And then, this next part is unclear -- is that mean the theory is not actually being tested and/or we just evaluate a subset of the described model (we think it's the latter! The actual test is a restricted subset of the verbal or conceptual model. ... what DOES restricted parameter space mean to theory? May be JA knows) 

\section*{Overcoming the chilling problem}
% JAOct2025: As of right now, this section seems unrelated to the unidentifiability problem. I think you have two points. 1. if chilling is real, its properties may be better understood by looking across different disciplines and combining observational/experimental evidence. 2. Is chilling really necessary for scientific progress?
Richer, more informative data from molecular biology studies and other approaches to identify chilling \citep{fouche2023transport,walde2024stable} hold the promise to shift chilling from an unobserved complex process to something we understand and can robustly forecast. Taking full advantage of this opportunity, however, would benefit from re-examining our concept our chilling by working across different disciplines to combine observational and experimental evidence, while also leaving opportunity to question the future utility of `chilling.' 

\subsection*{Model experimental and observational data together} 
Research on chilling could accelerate by working across what today are three fairly separated silos of model building for crops and other forecasting (usually called process-based models), experiments at the branch or whole-plant level, and molecular research. Currently, crop biologists, phenological process-based modelers of forest trees, molecular biologists and hardiness modelers all develop unique and rarely compared models of dormancy and budburst \citep[but see][]{kovaleskipreprint}, highlighting a major problem, but also a major opportunity. Synthesizing models---and their underlying biological understanding of chilling---across the many research fields developing chilling models today would help identify models that are equivalent.  % Uniting these models, first by fully defining their assumptions and conditions (e.g., what species are they designed for, what phenological or dormancy phase do they start at?), then comparing their predictions and pushing them to make different testable predictions would help build a unified model of chilling, with implications for better models of budburst, and cascading improvements in forecasts for crop yield and forest carbon sequestration. 

Synthesizing would hopefully driver new approaches that estimate chilling by fitting experimental and observational data together in one model. This is rarely (if ever) done, in part because of how differently they may be observed, including the challenging diversity of environmental conditions across these two data types (for example, many experiments apply cold temperatures in the dark and include extreme temperature differences, while photoperiod shifts each day in observational data and temperatures are more similar) but also because of separate modeling approaches. Researchers rarely if ever fit process-based models to new empirical data; instead they use so-called `statistical models' that often follow canonical treatment designs (e.g., ANOVA). Statistical models are usually far simpler, and make a suite of unstated assumptions that contradict the current understanding of chilling 
(see Fig. \ref{fig:multimodelexps}. Many of the original studies that led to the concept of chilling, however, were developed from datasets that created greater extremes in observational data---focusing on crops planted well outside their natural range (e.g., peaches in Florida and Israel) and bridged across observational and experimental studies more often \citep{erez1971,richardson1974}. % Improved statistical approaches should push the field of chilling back towards these foundational studies and fit observational and experimental data together, ideally in one model.  

Experiments bridging across methods may have the greatest opportunity to provide data that would truly challenge current models of chilling. Testing models across large environmental gradients in the field is one of the best ways to find out where models work---and where they fail. For example, molecular biologists tested vernalization models by through comparing predicted to observed flowering times in a common garden study across Europe \citep[][]{Wilczek:2009oa}, and supported the temperature-dependent growth model by testing its predictions of what happens when growth is altered but temperature is held constant \citep{zhao2020temperature}. Similar examples for challenging other models of chilling date back over 40 years to when many of the models used today were developed \citep{richardson1974,chuine2016,ospreebbms}, but could take place now. Models of chilling can make predictions under lower field chilling then test them using individuals planted beyond the range (either planting those individuals now or identifying such cases in forestry provenance trials or similar). Process-based modelers could also challenge their models more through more dramatic variation in biology, via experiments that include mutants or similar variants. 

% Fitting experimental and observational data should drive coherency in what chilling models perform best and reduce support for some models. This would limit the growing number of studies that have compared different process-based models on observational phenology data---where `natural' field conditions likely often satisfy chilling requirements for wild species \citep{basler2016evaluating,hufkens2018integrated}---to find the simplest models perform best. By adding more extreme experimental conditions we expect more complex models will perform well. Even if models cannot fit both data types together we suggest that new research should should include comparisons of the model performance on observational data and experimental data---models that cannot fit both data types should be flagged as indicating a potential problem with the model.  At the same time, datasets must provide the necessary information to fit chilling models (e.g., for experiments---what was the dormancy induction temperature, what as the thermo and photo-periodicity of each stage of the experiment). 


\subsection*{Is `chilling' necessary for scientific progress?}  % START HERE ... working on this section!

One important way to leverage new molecular insights for modeling is through new experiments designed to identify novel ways to more directly observe and measure chilling. % This means experiments  designed to identify markers of the underlying physiological stage and shifts between stages. 
Experiments testing for evidence of callose loss using the temperature treatments commonly applied in past studies \citep{ospreebbms} could be complemented by studies with other cellular and molecular markers \citep{yu2024building}. Testing these methods together alongside previous-used markers of dormancy shifts---including the often-used bioassay of high and rapid budburst at higher temperatures (indicating endo-dormancy release), and additional methods, such as weighing flower buds \citep{chuine2016} or tracing water reactivation into cells \citep{faust1991bound,Kalcsits2009,walde2024stable}---could help align both new and old methods. %Such approaches could help the field overcome its long reliance on rapid and high percent budburst under warm temperatures as indicating sufficient `chilling,' by providing more precise markers of physiological dormancy release \citep{fouche2023transport,walde2024stable}.

Because chilling is an unobserved process we argue that comparing methods to measure chilling should be a major priority for the field. This comparison will need to allow for the reality that different methods may measure different processes and, thus, terminology may need to adapt as well. As a first step, research could stop referring to treatments in experiments as `chilling' or `forcing,' or other terms that assume an underlying physiological state, and instead focus on the actual treatments (e.g., `cool temperatures before warm'). Currently, many experiments use the term `chilling' to refer to a treatment where researchers do not know the physiological phase \citep{flynn2018,ospreebbms}; for example, cuttings or buds from woody plants are often chilled at 5\degree C for 6 weeks in the dark in a `chilling' treatment, then transferred to warming `forcing' conditions. % 



%\vspace{10ex}
%{\sc Box: Why has progress on modeling stalled for decades?} % Latent non-identifiability.

%\begin{figure}[h!]
%\includegraphics[width=0.8\textwidth]{..//figures/boxfigures/figquickbox1.png}
%\includegraphics[width=0.8\textwidth]{..//figures/boxfigures/figquickbox2.png}
%\caption{Victor will update these figures.} 
%\label{fig:boxy}
%\end{figure}

% Most major models of chilling were developed over 40 years ago \citep{richardson1974,chuine2016} following several decades of new work---especially on peaches and other temperate tree fruit crops---and much debate \citep{dormtreeproc}. Since then many more models have been proposed \citep{luedeling2012chilling,chuine2016}, but models from the 1970s still often fit very well \citep{basler2016evaluating,chuine2016} and are widely used in major studies \citep[e.g.,][]{richardson1974,chuine2016,ospreebbms}. While we outline several reasons for this, including a disconnect between experimental, observational data and modeling approaches (e.g., process-based versus statistical), one of the main reasons for this stalled progress could be that most models---from the old to new ones---cannot actually estimate the parameters in them. That is, the models are fundamentally non-identifiable.

We argue that one of the main reasons for stalled progress on modeling chilling is that most models---from the old to new ones---cannot actually estimate the parameters in them. Taking a simple example of a chilling model with three parameters---the minimum temperature for chilling, the maximum temperature, and the accumulation needed---shows that there are multiple solutions. Considering just two of these possible solutions highlights how the temperature range trade-offs with the accumulation: if the temperature range is wide (lower minimum, higher maximum) then the accumulation required will be higher, while if the range is smaller, then the accumulation needed is lower. The full suite of possible solutions is effectively endless (and the trade-off between range and accumulation is not linear, as it depends on the full width of the range, but also its placement relative to 0). Further, this model is not actually one of only three parameters as two additional parameters were set as known (start day of accumulation was set at 1 September, and the endodormancy break date at 30 January) so that the model could even be fit using common algorithms. % Without setting these two parameters, the optimizing algorithm would not find solutions. 
This reality is present in every model of chilling, but it is rarely presented clearly. % https://github.com/lizzieinvancouver/chillingconcepts/issues/12

Researchers tacit approach to these major issues likely has contributed to the expansion of models over the last few decades without any clear advances. While various models have added complexity via the shape of the optimal temperature range for chilling, allowing accumulated chilling to be reduced, shifting the start date of chilling, and/or allowing chilling and forcing to act at once \citep{lued2009,gusewell2017,hanninen1990modelling,Kramer1994}, none of these have swept through the field. These new models of chilling have all added parameters, but none of the parameters added to chilling models in 40 years that have been successful enough to be added to all models of chilling. 

Being clear about model uncertainty, the full number of parameters and how well they fit would advance progress on chilling through multiple routes. First, it would help all researchers in the field recognize what is fundamentally unknown and thus focus more research in these areas. Second, it would highlight which parameters are most often fixed (effectively assumptions in the model) % vvdmSept23: this also relates to falsifiability and 'theory-laden' models
versus fit to data, and to which type of data. With this, more research could easily compare across models and datasets to give better overviews of what is known, assumed, or most often studied (i.e., what parameter model studies try to fit). Given the extensive list of proposed complexities to chilling models \citep{lamb1948effect,campbell1979,leinonen1996dependence,cook2005freezing,Man:2010aa,Jones:2012,Sonsteby:2014aa}, having simpler models (fewer parameters) that are routinely used to compare to more complex models, would likely help the field advance. % Currently, very simple models often outcompete more complex models, suggesting how we advance from these simpler models is also a problem \citep[e.g.,][]{basler2016evaluating}. 
% We argue that better models of chilling are possible given new insights alongside new approaches to how we model---and communicate---models of chilling. Models assumptions and identifiability need to be more clearly explained. This includes clearly defining chilling and endo-dormancy or similar transitions and how they are measured from data used to inform or parameterize models. As most current models of chilling appear non-identified---where all parameters cannot be clearly estimated from given data---more open review and discussion of this could greatly help compare models. 
Highlighting uncertainty in findings from experiments would also aid modeling studies to be more upfront about assumptions and limitations (Fig. \ref{fig:multimodelexps}).

% Highlighting uncertainty in findings from experiments would also aid modeling studies to be more upfront about assumptions and limitations. While experiments often assign treatments as `chilling' and `forcing,' most researchers know that the actual physiological transition may occur in the cool `chilling' treatment or much later in the warm `forcing' conditions, but fail to acknowledge it in their papers or statistical models \citep{ospreebbms,ospreenph2023}. While some experiments rely on a heuristic where budburst that is rapid and includes a high percentage of buds bursting indicates the physiological state associated with forcing (endo-dormancy) has started, the point when this threshold is crossed is rarely clear, and data showing it are often buried in supplements. These practices of hiding uncertainty in results and experimental designs that are effectively non-identifiable for the biological process under study likely contributes to confusion for researchers in the field and drives models with hidden assumptions. A coherent model requires researchers focused on experiments and modelers alike to embrace the current limited level of understanding in the field. % vvdmSept23: and data limitation?




\clearpage
\bibliographystyle{/Users/Lizzie/Documents/EndnoteRelated/Bibtex/styles/besjournals}
\section{References}
\bibliography{..//refs/bibforchillingconceptms}


\clearpage
%\section{Figures}

%\begin{figure}[h!]
%\includegraphics[width=0.7\textwidth]{..//figures/conceptModel/combinedModel.png}
%\caption{Aligning phenological models and molecular findings. Current phenological models of budbreak (top panel) assume two underlying states that lead to the observed process of budbreak: (1) endodormancy during which plants accumulate sufficient chilling to break endodormancy and transition to (2) ecodormancy, a period during which plants accumulate sufficient `forcing' (warm temperatures) to break bud (flower or leaf). Over 30 variants of this model exist, including those where the phases are sequential (darker shading) or occur in parallel (dark and light shading). Molecular work suggests callose (second panel), alongside Gibberellic acid (GA, third panel) and florigen related genes may all underlie these hypothesized phases.} 
%\label{fig:modelsketch}
%\end{figure}


%\begin{figure}[h!]
%\includegraphics[width=0.99\textwidth]{..//figures/quickcompare.pdf}
%\caption{Adding model diversity to experiments. Current experimental methods use limited data---often only on time to leafout (or flowering)---to estimate a hypothesized accumulation that triggers an unobserved event (currently often described as the release of endo-dormancy) which then leads to another accumulation that leads to leafout. Based on this model, many experiments alter `chilling' and `forcing' by varying the duration that cool temperatures are applied (`chilling') following by different warm temperatures (`forcing'). Diverging from this conceptual model, research often then fits simple linear models (though the accumulation model would be non-linear) with main effects of `chilling' and `forcing' treatments and their interaction (`chilling' $\times$ `forcing') to find a sub-additive effect of the two. This interaction is often interpreted to mean that longer cool temperatures (`chilling') lead to a greater requirement of `forcing,' but is rarely if ever compared to alternative models. Here, using data from \citet{walde2022higher} for \emph{Quercus robur}, we fit a non-linear model and compare the common model where cool and warm treatments interact with, such that longer cool temperatures mean more warm temperatures are required for leafout, with a model where longer cool temperatures change the start date of forcing (effectively, were the date of endo-dormancy break is shifted by cool temperatures, not the required warm accumulation). We can also add something about which fits better if I work on that .... This latter model is arguably more in line with the current biological model but rarely fit.} 
%\label{fig:multimodelexps}
%\end{figure}

\end{document}

\begin{figure}[h!]
\includegraphics[width=0.9\textwidth]{..//figures/biochemicalModelSketch.png}
\caption{Draft visualization of molecular pathways identified in vernalization and chilling; if we want to keep this, we can update to clarify which have been found in woody species perhaps and match the text a little more. ADD to caption: In the first phase---endo-dormancy---plants are accumulating `chilling' and cannot respond to external warm periods  \citep[thus the `endo' part of the term,][]{chuine2016,lundell2020}. Once they have accumulated the appropriate amount of `chilling' they are `eco-dormant,' and will leaf or flower in response to a sufficient thermal sum (often called forcing, or `growing degree days' in some models). } 
\label{fig:molecular}
\end{figure}

{\sc Box: What we know matters to chilling, and what might matter} % and why we keep going around in circles on negative temperatures?

Studies on how cool winter temperatures affect budburst have identified numerous possible factors that appear to moderate `chilling.' Certainly time and cool temperatures affect budburst timing. Teasing out the effect of time versus temperature has proven difficult, but plants and cuttings (clipped ends of tree branches with one or more bud) exposed for the same amount of time to different constant cool temperatures budburst at different rates and percentages \citep{weinberger,ospreenph2023}. Which temperatures are most effective at providing chilling---estimated as those that lead to either (or both) the highest percent or fastest budburst after exposure to warm (often 20\degree C) temperatures---is a common question addressed in experiments, with different experiments providing different answers \citep{vitasselev,baum2021}. 

This has led to debate recently on whether sub-zero temperatures can provide chilling as one recent study appeared to show \citep{baum2021}, supporting previous research \citep{Jones:2012,Sonsteby:2014aa}, but in contrast to other studies \citep{lamb1948effect,cook2005freezing,Man:2010aa}. These studies, however, vary in a number of additional factors including the species and populations they study, the full conditions under which cool temperatures were applied etc.. This debate highlights the complexity of identifying the dominant controllers on chilling when so many factors vary across studies, and so many factors appear to moderate effects. In addition to the temperature range, temperature variability also appears to play a role CITES. The role of sub-zero temperatures and temperature variability also matter strongly to plant hardiness---what level of cold temperatures plants can withstand without damage, which is induced alongside dormancy in the fall for most temperature species. Recent work by \citet{kovaleskipreprint} has reignited an old debate about whether hardiness influences dormancy CITES---or vice versa. 

% Maybe cut the below? Seems annoying nanny-like text (says the person who wrote it)
These debates---about the importance of sub-zero temperatures and plant hardiness to dormancy and chilling---are not new, and also highlight gaps in the approach many biologists have taken to studying chilling in an era where how well we understand it has major implications due to anthropogenic climate change. Some recent papers  ask similar questions to those of decades ago \citep[e.g.,][]{lamb1948effect,Man:2010aa,baum2021}. While this highlights that such debates may not have been fully settled it can also slow progress. Because of this, we suggest a major need is to build and share databases of all relevant studies---across molecular, crop, and forest biology. Such databases should include information on factors that appear to affect chilling, including dormancy induction temperature, species, provenance location  \citep[as different populations may need different chilling][]{campbell1979,leinonen1996dependence}, thermo- and photo-periodicity of all applied treatments, and any hardiness information.  % Maybe ...  We provide a start to this in the supplement?

% While better documenting the diversity of species and treatments applied in chilling-related studies to date will help, progress may require a greater focus on solving the problem for one species ..... 


\vspace{5ex}
{\sc Possible other Box: The problem with how we measure endo-dormancy release currently} % could also include that we don't measure it much

Traditionally, experiments to test for effective chilling temperature ranges in woody species have used a heuristic where budburst that is rapid and includes a high percentage of buds bursting indicates endo-dormancy release has happened. This method has the benefit of being a useful bioassay---testing if plants can budburst as a metric of an endo-dormancy release (indeed, this method and the term were developed together, CITES)---but has disadvantages in that the method, which works well for some stone fruits, often appears much messier in other species. Because the method relies on fitting a hypothesized curve (or simple threshold) to set a date or level of for endo-dormancy break, how well it performs for many species is often not clear CITES, and thus whether it is an accurate bioassay is similarly unclear. For this reason, others have argued for other methods, such as weighing flower buds \citep{chuine2016}CITES or tracing water reactivation into cells \citep{faust1991bound,Kalcsits2009}. 

% Bad things without a home ...
% And then there are attempts to estimate chilling using observational data in crops (but often when planted outside range) and forest trees ... and basically always based on assumptions from existing models and/or experiments (peaches again).
