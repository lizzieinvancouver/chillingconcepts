\documentclass[11pt]{article}
\usepackage[top=1.00in, bottom=1.0in, left=1in, right=1in]{geometry}
\renewcommand{\baselinestretch}{1.1}
\usepackage{graphicx}
\usepackage{natbib}
\usepackage{amsmath}
\usepackage{gensymb}
\usepackage{parskip}
\usepackage{xcolor}
\usepackage[normalem]{ulem}

\def\labelitemi{--}
\parindent=0pt

\begin{document}
\renewcommand{\refname}{\CHead{}}


\title{The problem and promise of \\ modeling plant leafout under warmer winters} %  How overconfident models of tree leafout have stalled progress... Infinite models of leafout given warming winters...? % I don't love this title.
\author{E.M. Wolkovich$^1$, Justin Ngo$^1$, Victor van der Meersch$^1$, Jonathan Auerbach$^2$} % also possibly: , Ron Guy ... Frederik Baumgarten$^1$ though he has been so MIA, I am not sure if he is interested and this paper is unlike the one we proposed to work on together. 
\date{\today}
\maketitle

$^1$ Forest \& Conservation Sciences, Faculty of Forestry, University of British Columbia, 2424 Main Mall, Vancouver, BC V6T 1Z4, Canada\\
$^2$ Department of Statistics, George Mason University, 4511 Patriot Circle, Fairfax, VA 22030, United States\\

% As of 22 Nov 2025, I have notes on next steps in _dothiChilling.md

\begin{abstract} % 215 words -- should cut 15 words
How warmer winters affect plant leafout has major implications for carbon storage and ecosystem stability. Recent evidence shows the advance of spring leafout has slowed as anthropogenic climate change has increased, and attempts to explain the slowed advance has reignited a decades-old debate over winter `chillling'---the concept that woody temperate plants require sufficient winter cool temperatures to budburst each spring. The problem with chilling is that it is largely used as a placeholder theory:  it does not correspond with an observable event, is not falsifiable, and thus is of limited use for advancing scientific understanding. In this paper, we review how current research is built on the theory of chilling, which, due in part to limited data, has led to the proliferation of untestable models that are consistent with a wide range of forecasts and provide little insight into actual plant behavior. We argue that the idea of chilling as a scientific theory must be reevaluated like past theoretical constructs of limited scientific value, such as `dormin' or the `ether,'  until which time researchers will be slow to incorporate new insights from molecular biology. This can be done by building better models by looking for the necessary insight from molecular biology to fill in the details and designing better experiments to test those theories.
% We conclude that integrating new insights and approaches from molecular biology could better define the mechanism of chilling leading to models that make falsifiable predictions. Exposing the gaps in current mechanistic and statistical models at the same time new experiments yield richer, more informative data could greatly reduce the number of competing models of chilling that we have today---advancing our fundamental understanding of plant dormancy---and lead to much improved forecasts from crops to forest trees.  
\end{abstract}
% ... Such extreme variability of spring phenology predicted from current models suggests fundamental and major gaps in our understanding. Indeed, new research suggests major flaws in these models as currently applied, including multiple papers now suggesting estimates of `chilling' could easily be artifacts of poor models or correlated observational climate data \citep{decsens,gao2024}. 

% Main text: 3800; Box 1: 500.
Plant leafout in the spring has shifted weeks earlier in many regions due to warming from anthropogenic climate change with consequences for a wide range of ecosystem services \citep{keenan2014net,ipcc2022}. Which underlying processes drive this trend, however, is debated, as recent research suggests winter warming may slow or stall further advances \citep{fu2015,piao2017}. Such reports focus on a two-step model of leafout where plants first require cool winter temperatures, often called `chilling,' before they can accumulate enough warm temperatures---`forcing'---to leafout each year. This model of chilling is only one of many models proposed since the concept was first introduced \citep[][]{basler2016evaluating,hufkens2018integrated}. But, while forcing ends in an observable event such as leafout, the day plants achieve their chilling requirements coincides with no known measurable event. 

Current models of chilling are consistent with a diverse suite of possible future leafout. Chilling in the same location for the same plant under different models can easily forecast significantly increased or greatly reduced chilling and thus greatly advanced or slowed leafout \citep{guy2014,chuine2016}. This variability often occurs within any one model of chilling, given different parameter values (Figure \ref{fig:nonidentifyme}). Such extreme variability of spring phenology predicted from current models highlights fundamental gaps in our understanding. While decades of experimental evidence suggest that winter temperatures impact leafout \citep{charrier2015,baum2021} new research has highlighted major flaws in these models as currently applied, including multiple papers now suggesting estimates of `chilling' could easily be artifacts of poor models or correlated observational climate data \citep{decsens,gao2024}. 

Now appears an opportune time to address the problems in our concept of chilling in models of leafout. Because accurate forecasts of spring phenology are critical for carbon sequestration, many crops and a number of other services important to humans, there is widespread interest in improving chilling models \citep{Luedeling2015Acta,chuine2016}. At the same time, new results from molecular and cellular studies of dormancy are providing new insights into when and how chilling works \citep{pan2023epigenetic,zhu2021cold}, which could rule in---or out---some of the myriad current models. % which the myriad models could be tested against. 
Here, we review the concept of chilling, its origins and potential problems, as well as new opportunities for major advances and how shifting current practices could accelerate progress. We argue that revisiting the chilling mechanism could help build better models and design better experiments. % In particular, we discuss how new insights from molecular and cellular studies combined with modern approaches to building biological models could revolutionize our understanding of chilling. 

\section*{What is chilling?}
% JAOct2025:  Great section, would not cut.

%addbook
How plants in temperature-limited systems avoid leafout during warm spells in the winter has long been debated by plant biologists \citep[e.g.,][]{lamb1948effect,weinberger}. Most work to date has focused on the idea that plants enter some form of dormancy in the fall, which is then released before warm temperatures in the spring begin. This idea hypothesizes that the slow accumulation of cool temperatures---or chilling---over the winter extends through periods of short warm spells in the winter and thus prevents leafout before spring. The term `chilling' is now used across numerous fields in plant biology to refer to a process where dormant buds exposed to cool temperatures accelerate a phenological event that later occurs after warm temperatures. 

Much of our fundamental understanding of chilling comes from studies on temperate woody fruit crops where chilling can be critical to yield. Peach trees planted into warmer climates well outside their range often have extremely low fruitset because most flower buds do not burst \citep{weinberger,overcash1955effects,erez1971improved}. Initial studies of this phenomenon with related experiments---where cut ends of dormant branches (cuttings) exposed to cooler temperatures in chambers burst more fully and more quickly---underlies most of the models of chilling used today for crops and wild tree species \citep[][]{weinberger,ospreebbms}. %, but these have rarely---if ever---has been well tested. 

Focused on how chilling can accelerate events, researchers have calculated `chilling' required for leafout of forest trees from cutting experiments similar to those used for peaches \citep[reviewed in][]{ospreebbms}, ground observations of budburst \citep{Luedeling2009}, and satellite measures of greenup \citep{kaduk2011predicting}. These estimates rely on phenological models that have become critical across a suite of fields, from climatology, where modeling how vegetation on the land surface responds to anthropogenic climate change affects carbon storage, to crop biology, where estimated chill units guide growers in which specific cultivar to plant, and has led to the cross-disciplinary field of phenology \citep{Schwartz:1994he,Cleland:2007or,pmp}. % A suite of studies on forest trees have also estimated chilling using cuttings of tree branches during the winter in experiments where `chilling' is manipulated either in controlled environment chambers or through the sequential removal of cuttings CITES. While most of these studies focus on leafout, a similar suite of studies on crops has focused on flower burst. 

Alongside these more macro-scale studies of chilling, molecular approaches have also examined chilling. Many studies have focused on vernalization---cool temperatures required for flowering \citep{kim2009vernalization}---in \emph{Arabidopsis thaliana}, with studies in woody species, especially \emph{Populus} examining chilling before budbreak \citep[][]{azeez2021early,cai2024molecular}. These studies generally use controlled temperatures to vary the hypothesized amount of chilling then examine molecular and cellular responses \citep[e.g.,][]{pan2021aba,azeez2021early,cai2024molecular}.
% Versions of plants with knock-outs of specific genes have helped identify genes that underlie dormancy and other processes related to chilling \citep{songstad2017genome}. 

Today these studies have led to over 30 basic models where accumulated chilling releases plants from dormancy \citep[][]{basler2016evaluating,hufkens2018integrated}. Though early debates considered whether plants were truly dormant or only growing slow \citep[`dormancy' or `rest' versus `quiescence';][]{considine2016language}, today most research assumes a model with two phases of dormancy: endodormany---where chilling occurs---and a period after but before the observed event, called ecodormancy (Figure \ref{fig:modelsketch}).  In most models, chilling can only be accumulated under certain temperatures---traditionally above zero but below 10\degree C---with certain temperatures 
being optimal for the most rapid accumulation of `chill units,' where some unknown sum of chill units breaks endo-dormancy. Which temperatures are most effective at providing chilling is a common question addressed in experiments, with different experiments providing different answers \citep{vitasselev,baum2021}, and some suggesting the threshold likely varies by species. 

Variation in chill requirements by and within species yields hundreds of variants of the basic models \citep[][]{basler2016evaluating,hufkens2018integrated}. Complexity comes from the hypothesized diversity of these temperature thresholds and sums across species and populations. Most assume different species require different sums of chill units, and may have different lower, upper and optimal chill temperatures. Within species, populations may require different sums of chill units, with populations in more mild climates---where warm interruptions in the winter are more common---requiring more chill units than those in areas with cold winters, where temperatures rarely rise above zero before spring \citep{campbell1979,leinonen1996dependence}.  
%vvdm25oct2025: maybe you could cut the next sentence related to GDD?
% This mirrors growing degree days in many ways, where a lower temperature (e.g., 5 or 10\degree C base temperature) is too cold for forcing units to accumulate and plants need some total sum of such units to leafout or flower, but has added complexity given the importance of both the lower and upper temperature thresholds for `chill units' \citep[whereas growing degree day models can often ignore the upper threshold, estimated at 25\degree C or above, as it is rarely reached in natural spring conditions,][]{mcmaster1997growing,li2021comparisons}. % This has led to debate recently on whether sub-zero temperatures can provide chilling as one recent study appeared to show \citep{baum2021}, supporting previous research \citep{Jones:2012,Sonsteby:2014aa}, but in contrast to other studies \citep{lamb1948effect,cook2005freezing,Man:2010aa}. These studies, however, vary in a number of additional factors including the species and populations they study. 

\section*{The problem with chilling}

\subsection*{An unobserved process}

Chilling is a latent, unobserved process. Typically, %little progress towards elucidating this process over the past 40 years \citep{chuine2016,hanninen2019experiments}. 
%Most models of chilling used today for estimating budburst in crops, land surface modeling etc. are process-based, meaning they attempt to represent the underlying mechanistic process. 
%This underling process and the models of it were developed in the 1970s \citep{richardson1974} following several decades of new work---especially on peaches and other temperate tree fruit crops---and much debate \citep{dormtreeproc}. Since then many more models have been proposed \citep{luedeling2012chilling,chuine2016}, but the original models still fit most data well compared to newer models \citep{basler2016evaluating,chuine2016} and are widely used in major studies \citep[e.g.,][]{richardson1974,chuine2016,ospreebbms}. One of the main reasons for this stalled progress is that almost all models have more unobserved components of the mechanistic process than they can fit with current data. 
%Thus, cannot actually estimate the parameters in them, and are fundamentally non-identifiable (see Box: Why has progress on modeling stalled for decades?).
% Experiments and models focused on chilling describe an unobserved process \citep{chuine2016}, with critical differences  between the two approaches. 
%
% Current models use limited data to estimate a hypothesized accumulation that triggers an unobserved event---currently often described as the release of endo-dormancy---which then leads to another accumulation that leads to leafout or flowering. 
`chilling' describes the physiological phase (endo-dormancy) in which a plant experiences environmental temperatures that induce progress towards the next physiological phase (eco-dormancy) that ends in budburst (Figure \ref{fig:modelsketch}). The problem is that the transition from endo- to eco-dormancy corresponds with no clearly measurable phenomenon, and thus the properties of each period cannot be determined from the observed budburst, even under experimental conditions in which the temperature can be manipulated. The parameters of a model governing this system are said to be underdetermined or unidentified. 

For example, consider a simple model in which one would need to estimate the minimum and maximum temperatures that allow chilling to accumulate (two parameters) and the total sum of those temperature units needed to trigger a shift into the next physiological phase (often called, `endo-dormancy break,' for one additional parameter). Models then need to estimate when plants start and stop accumulating (two more parameters). An experiment that raises temperatures and observes an earlier leafout could be explained by more rapid chilling accumulation leading to earlier endo-dormancy break, more rapid ecodormancy break, or an almost limitless mix of the two (Figure \ref{fig:nonidentifyme}). % Even in some of the simplest models, estimates of what temperatures accumulate `chilling' and how much chilling is required to shift physiological stages often occupy multiple divergent options. A common example of this is one outcome where the range of temperatures that allow chilling to accumulate is wide and thus the threshold amount of chilling large, or an outcome where the range is smaller and, thus, the threshold lower. Yet researchers often only present one of these outcomes \citep{chuine2016}, versus showing the full range of possible values---that is, the full uncertainty. 

To address this, models often assign the start date of the endo-dormancy as known (e.g., starting 1 September in the northern hemisphere) and rely on assumptions to set the end date (Figure \ref{fig:nonidentifyme}). A common and widely-used assumption, developed by early work on peaches, is that high and rapid budburst (leaf or flower) is evidence that chilling has been met \citep[i.e. endo-dormancy has ended][]{erez1971}, but is rarely evaluated for other taxa or populations. This approach of assigning some unknown parameters as known has the benefit of avoiding adding more unknown model parameters, but has led researchers to be overly confident in a model of chilling where more is actually unobserved and unknown than acknowledged. 

\subsection*{Hidden assumptions drive more complicated models}

Hidden assumptions and numerous parameters can easily drive diverging models. Even if we assume that high and rapid percent budburst signals sufficient chilling, most models today include parameters that cannot be uniquely identified with current data, and explains why one model can often predict a huge variety of leafout possibilities (Figure \ref{fig:nonidentifyme}). Researchers tacit approach to these major issues likely has contributed to the expansion of models over the last few decades without any clear advances. While various models have added complexity via the shape of the optimal temperature range for chilling, allowing accumulated chilling to be reduced, shifting the start date of chilling, and/or allowing chilling and forcing to act at once \citep{lued2009,gusewell2017,hanninen1990modelling,Kramer1994}, none of these have swept through the field. % These new models of chilling have all added parameters, but none of the parameters added to chilling models in 40 years that have been successful enough to be added to all models of chilling. 

% Given experiments and models have suggested many variants on a more complicated model of chilling---for example minimum, maximum and optimal temperatures, or high temperatures that reduce previous accumulation \citep[Figure \ref{fig:modelsketch}, and see][]{lued2011,luedeling2012chilling,chuine2016}---current data are relatively uninformative to try to estimate all the parameters the models include. 
As experiments and models have suggested a more complicated model of chilling, data have remained relatively uninformative to try to estimate all the complexities the models include. Further, recent models have often relied on even less data \citep{hanninen2019experiments}. Many current methods use only observational data of the timing of leafout (or flowering) to attempt to estimate a model of chilling for different species or locations and project it forward to understand effects of anthropogenic climate change \citep{lued2011,luedeling2012chilling,gao2024}. Perhaps not surprisingly then, which model is deemed best varies strongly by method and approach \citep{Caffarra:2011qf,basler2016evaluating,hufkens2018integrated}, with no clear pattern. 

\subsection*{Unclear aims}

The flood of current models of budburst with similar levels of predictive accuracy is not necessarily a problem, depending on the aim. Many models of chilling are used for helping guide crop and tree planting---preventing planting too far outside the range of climates with sufficient chilling for a species, population, or cultivar. For this aim, models need to translate climate into relevant chilling and forcing values and define appropriate thresholds, but how exactly each model defines chilling and forcing is immaterial if the predictions are accurate enough. In such cases moving more towards machine learning models---which emphasize predictive accuracy often through black box approaches---may provide improved models and more clearly acknowledge that the underlying process is obscure \citep{khodadadzadeh}. In contrast, if the aim is biological understanding, then the current number of highly contrasting but similarly predictive models suggests a problem.
The most predictively accurate models may then not be favored if they fail to represent our  

If the aim is to understand the biological process that underlies the concept of chilling, then the specification of the model---what parameters it includes, and how well they can be estimated from the data---is important. In these cases, the model should mathematically represent the current biological understanding, and we would likely expect fewer models than the current proliferation. This aim also suggests comparing models through direct tests of how accurate they represent our understanding of plant biology, and not just their predictive accuracies across a range of climate regimes. Such models may also be predictively accurate, especially in cases beyond current climate regimes, but the primary goal is that they represent and help test plant biology.  For this aim, new models are unlikely to be enough. Instead the field will likely need new data and approaches target the major problems facing models of chilling. %---if the field is more open about those issues and open to new approaches

% JA: In this section, it sounds to me like you're saying how can molecular biology refine the theory of chilling such that science can progress. Logically, that seems to me to fall in the next section, "Overcoming the chilling problem." In this section, I was expecting that you might bridge to this idea by point out that if the goal is simply to predict when the peach trees will bloom given observed chilling and forcing, the incorrectly specified models are probably fine. If the goal is biological understanding and then extrapolation/anticipation of how ecosystems will be affected, the parameterization matters. Molecular biology (and machine learning/AI for that matter) may provide new ways to test these theories, but they also add more parameters/levers and do not automatically solve the problem unless chilling becomes observable/testable. i.e. we have a chicken and egg problem: if we observed chill, we could identify the genes that code for it, and if we knew the genes that code for chill, we could measure/observe chill. Progress requires that we move beyond chilling to something more scientific, e.g. your callose example, which again seems like "overcoming." 


\section*{Overcoming the chilling problem}
% JAOct2025: As of right now, this section seems unrelated to the unidentifiability problem. I think you have two points. 1. if chilling is real, its properties may be better understood by looking across different disciplines and combining observational/experimental evidence. 2. Is chilling really necessary for scientific progress?
Richer, more informative data from molecular biology studies and other approaches \citep{fouche2023transport,walde2024stable} hold the promise to shift chilling from an unobserved complex process to something we understand and can robustly forecast. Taking full advantage of this opportunity, however, would benefit from re-examining the concept our chilling across different disciplines to combine observational and experimental evidence, while providing the opportunity to question the future utility of `chilling.' 

\subsection*{Leverage new molecular insights}

% Molecular insights have long helped crop and forest tree models of chilling CITES, but recent findings appear especially promising to build a more unified and coherent model. 
New molecular research provides hope that the transition from endo- to eco-dormancy is measurable. Molecular insights have long contributed to crop and forest tree models of chilling \citep{chuinearees}. Decades of work on vernalization have outlined the pathways---and genes---that lead to flowering only after winter's cool temperatures in biennial (herbaceous) populations of \emph{Arabidopsis thaliana} \citep[Figure \ref{fig:modelsketch},][]{Wilczek:2009oa,kim2009vernalization}. Research has linked some of these pathways to similar ones in woody species, and have also highlighted the sugar callose (1,3-$\beta$-{\sc D}-glucan) as potentially pivotal for chilling \citep{vanderschoot2014,pan2021aba}. Multiple studies across multiple species have now shown that (1) lower temperatures appear to degrade callose and (2) the release/loss of callose appears to re-start cell-to-cell communication before budburst \citep{vanderschoot2014}. Taken together, these results suggest the loss of callose---generally degraded through 1,3-$\beta$-glucanases (a group of enzymes)---may be an indicator of endo-dormancy release, though other factors, such as  phytohormones, also often change at the same time \citep[][]{tylewicz2018photoperiodic,pan2021aba}, and may provide a similar observable signal of endo-dormancy release \citep{rinne2018,andre2022populus}. 
% New cellular and molecular results could potentially reduce this complexity, and help narrow the set of possible models.  We argue that two new major insights from molecular studies---the importance of callose and temperature-dependent growth---could limit the chilling models considered today, and reshape the experiments tree biologists use to determine `chilling.' 

% MOVE this text to a figure caption maybe?
% The hypothesis that the sugar callose may play a pivotal role in bud endodormancy has been suggested for over a decade \citep{rinne2011}, but recent studies have yielded increasing support for this hypothesis \citep{vanderschoot2014,pan2023epigenetic}. Callose (1,3-$\beta$-{\sc D}-glucan), which is synthesized for various reasons in plants (e.g., at the site of infection to prevent pathogens), also appears to block plasmodesmata in dormant buds. 

If callose is a major controller of endo-dormancy and its release, chilling models could be limited to those that match the idea of callose degradation---meaning models that include a temperature range over which the enzyme is active (Figure \ref{fig:modelsketch}). In contrast, models using simple temperature thresholds (e.g., all hours below -5\degree C equally allow chilling) would appear less biologically accurate, as enzymes generally do not work over such a wide range of temperatures. 

Other new molecular insights similarly suggest that such simplified temperature metrics used in many chilling models may not map to molecular realities. For example, new work on how slow growth itself may act  a `long-term thermosensor' \citep{zhao2020temperature} 
adds to an increasing number of molecular studies that suggest plants integrate long-term thermo-sensing in the winter alongside responses to short-term temperatures \citep{antoniou2021feeling,Satake2022}. The best models of leafout may thus need to integrate across multiple timescales. While this could easily complicate models already burdened with complexity, new insights from molecular biology could rule out models by focusing on new experiments and modeling approaches. 
% especially given how little of the dormancy process we currently observe.  


\subsection*{Model experimental and observational data together} 
Research on chilling could accelerate by working across what today are three fairly separated silos: model building for crops and other forecasting needs (usually called process-based models), experiments at the branch or whole-plant level, and molecular research. Currently, crop biologists, phenological process-based modelers of forest trees, molecular biologists and hardiness modelers
all develop unique and rarely compared models of dormancy and budburst \citep[but see][]{kovaleskipreprint}, highlighting a major problem, but also a major opportunity. Synthesizing and benchmarking models---and their underlying biological understanding of chilling---across the many research fields developing chilling models today would help identify models that are equivalent and/or perform especially poorly and could begin to discard some models.  % Uniting these models, first by fully defining their assumptions and conditions (e.g., what species are they designed for, what phenological or dormancy phase do they start at?), then comparing their predictions and pushing them to make different testable predictions would help build a unified model of chilling, with implications for better models of budburst, and cascading improvements in forecasts for crop yield and forest carbon sequestration. 

Comparing models more directly would hopefully drive new approaches that estimate chilling by fitting experimental and observational data together in one model. This is rarely (if ever) done, in part because of how differently they may be observed, including the challenging diversity of environmental conditions across these two data types (for example, many experiments apply cold temperatures in the dark and include extreme temperature differences, while in observational data photoperiod shifts each day and temperatures are more similar) but also because of separate modeling approaches. Researchers rarely if ever fit process-based models to new empirical data; instead they use so-called `statistical models' that often follow canonical treatment designs (e.g., ANOVA). Statistical models are usually far simpler, and make a suite of unstated assumptions that contradict the current understanding of chilling 
(see Figure \ref{fig:multimodelexps}). Many of the original studies that led to the concept of chilling, however, bridged across observational and experimental studies more often and leveraged datasets that created greater extremes in observational data---by focusing on crops planted well outside their natural range \citep[e.g., peaches in Florida and Israel, ][]{erez1971,richardson1974}. These approaches did this in part by more explicitly mapping from biological processes to statistical models, making it easier to use the models to define new tests to challenge the biological theory. . % Improved statistical approaches should push the field of chilling back towards these foundational studies and fit observational and experimental data together, ideally in one model.  

Experiments bridging across methods may have the greatest opportunity to provide data that would truly challenge current models of chilling. Testing models across large environmental gradients in the field is one of the best ways to find out where models work---and where they fail. For example, molecular biologists tested vernalization models by through comparing predicted to observed flowering times in a common garden study across Europe \citep[][]{Wilczek:2009oa}, and supported the temperature-dependent growth model by testing its predictions of what happens when growth is altered but temperature is held constant \citep{zhao2020temperature}. Similar examples for challenging other models of chilling date back over 40 years to when many of the models used today were developed \citep{richardson1974,chuine2016,ospreebbms}, but could take place now. Models of chilling can make predictions under lower field chilling then test them using individuals planted beyond the range (either planting those individuals now or identifying such cases in forestry provenance trials or similar). Process-based modelers could also challenge their models more through more dramatic variation in biology, via experiments that include genetic mutants or similar variants. 

% Fitting experimental and observational data should drive coherency in what chilling models perform best and reduce support for some models. This would limit the growing number of studies that have compared different process-based models on observational phenology data---where `natural' field conditions likely often satisfy chilling requirements for wild species \citep{basler2016evaluating,hufkens2018integrated}---to find the simplest models perform best. By adding more extreme experimental conditions we expect more complex models will perform well. Even if models cannot fit both data types together we suggest that new research should should include comparisons of the model performance on observational data and experimental data---models that cannot fit both data types should be flagged as indicating a potential problem with the model.  At the same time, datasets must provide the necessary information to fit chilling models (e.g., for experiments---what was the dormancy induction temperature, what as the thermo and photo-periodicity of each stage of the experiment). 


\subsection*{Is the concept of `chilling' necessary for scientific progress?}  

Results from new molecular experiments designed to identify novel ways to directly observe and measure chilling could upend the current concept of `chilling.' As increasing evidence suggests the concept aligns with multiple shifts at the cellular and molecular level (Figure \ref{fig:modelsketch}), researchers across disciplines will need to consider how to continue using the term `chilling.' One approach is to try to align previous and current definitions through new studies,  % Experiments testing for evidence of callose loss using the temperature treatments commonly applied in past studies \citep{ospreebbms} could be complemented by studies with other cellular and molecular markers \citep{yu2024building}. % used to cite {fig:molecular} above
for example, by testing for callose loss \citep{yu2024building} alongside previous-used markers of dormancy shifts---including the often-used bioassay of high and rapid budburst at higher temperatures (indicating endo-dormancy release), and additional methods, such as weighing flower buds \citep{chuine2016} or tracing water reactivation into cells \citep{faust1991bound,Kalcsits2009,walde2024stable}. %Such approaches could help the field overcome its long reliance on rapid and high percent budburst under warm temperatures as indicating sufficient `chilling,' by providing more precise markers of physiological dormancy release \citep{fouche2023transport,walde2024stable}.
Another approach is to work towards new terms that more clearly align with new measurements and insights. 

Because chilling is an unobserved process, it may link eventually to one measurable compound or regulator (similar to florigen or auxin) or to a more complicated biology, making the term less useful \citep{aksenova2006florigen}. Current results suggests the latter, and already the term `chilling' is used to mean different things. For example, while we focus here on the term as an accumulation process during an unobserved physiological phase (endo-dormancy), many experiments currently use the term `chilling' to refer to a treatment where researchers do not know the physiological phase \citep[for example, cuttings or buds from woody plants are often chilled at 5\degree C for 6 weeks in the dark in a `chilling' treatment, then transferred to warming `forcing' conditions,][]{flynn2018,ospreebbms}.  

Diverging definitions suggest that the best path for `chilling' may be to follow that of other terms modern plant biologists may not remember, such as dormin or anthesin \citep[hypothesized compounds to induce dormancy and floral anthesis, respectively,][]{aksenova2006florigen,dorffling2015discovery}, and let the term fade into obsolescence. Models for forecasting and new experiments at the branch and whole plant level would then focus on modeling and measuring genes, regulators, hormones and other compounds that emerged from a history of research on 'chilling,' but use other terms and concepts. Chill unit estimates for widely planted forest trees and specific varieties of crops would likely hold on longer. They too, however, may shift to new estimates or units based on better measures, especially as increasing anthropogenic climate change makes getting the mechanistic biology of plant development right for human adaptation more important. 
% “florigen” is in fact the protein encoded by the gene FLOWERING LOCUS T (FT) .. The term “florigen” was coined in 1936 by the Soviet-Armenian plant physiologist Mikhail Chailakhyan.

% One important way to leverage new molecular insights for modeling is through new experiments designed to identify novel ways to more directly observe and measure chilling. % This means experiments  designed to identify markers of the underlying physiological stage and shifts between stages. 

% Because chilling is an unobserved process we argue that comparing methods to measure chilling should be a major priority for the field. This comparison will need to allow for the reality that different methods may measure different processes and, thus, terminology may need to adapt as well. 


\clearpage
\bibliographystyle{/Users/Lizzie/Documents/EndnoteRelated/Bibtex/styles/besjournals}
\section{References}
\bibliography{..//refs/bibforchillingconceptms}


\clearpage
\section{Figures}

\begin{figure}[h!]
\includegraphics[width=0.8\textwidth]{..//figures/boxfigures/figquickbox1.png}
\includegraphics[width=0.8\textwidth]{..//figures/boxfigures/figquickbox2.png}
\caption{A major problem with current models of chilling is that they cannot uniquely estimate the most important aspects of chilling. We show here a simple example of a chilling model with three parameters---the minimum temperature for chilling, the maximum temperature, and the accumulation needed---shows that there are multiple solutions. Considering just two of these possible solutions highlights how the temperature range trade-offs with the accumulation: if the temperature range is wide (lower minimum, higher maximum) then the accumulation required will be higher, while if the range is smaller, then the accumulation needed is lower. The full suite of possible solutions is effectively endless (and the trade-off between range and accumulation is not linear, as it depends on the full width of the range, but also its placement relative to 0). Further, this model is not actually one of only three parameters as two additional parameters were set as known (start day of accumulation was set at 1 September, and the endodormancy break date at 30 January) so that the model could even be fit using common algorithms. This reality is present in every model of chilling, but rarely discussed and often not even mentioned.} % https://github.com/lizzieinvancouver/chillingconcepts/issues/12} 
\label{fig:nonidentifyme}
\end{figure}

\begin{figure}[h!]
\includegraphics[width=0.7\textwidth]{..//figures/conceptModel/combinedModel.png}
\caption{Aligning phenological models and molecular findings. Current phenological models of budbreak (top panel) assume two underlying states that lead to the observed process of budbreak: (1) endodormancy during which plants accumulate sufficient chilling to break endodormancy and transition to (2) ecodormancy, a period during which plants accumulate sufficient `forcing' (warm temperatures) to break bud (flower or leaf). Over 30 variants of this model exist, including those where the phases are sequential (darker shading) or occur in parallel (dark and light shading). Molecular work suggests callose (second panel), alongside Gibberellic acid (GA, third panel) and florigen related genes may all underlie these hypothesized phases.} 
\label{fig:modelsketch}
\end{figure}



\begin{figure}[h!]
\includegraphics[width=0.99\textwidth]{..//figures/quickcompare.pdf}
\caption{Adding model diversity to experiments. Current experimental methods use limited data---often only on time to leafout (or flowering)---to estimate a hypothesized accumulation that triggers an unobserved event (currently often described as the release of endo-dormancy) which then leads to another accumulation that leads to leafout. Based on this model, many experiments alter `chilling' and `forcing' by varying the duration that cool temperatures are applied (`chilling') following by different warm temperatures (`forcing'). Diverging from this conceptual model, research often then fits simple linear models (though the accumulation model would be non-linear) with main effects of `chilling' and `forcing' treatments and their interaction (`chilling' $\times$ `forcing') to find a sub-additive effect of the two. This interaction is often interpreted to mean that longer cool temperatures (`chilling') lead to a greater requirement of `forcing,' but is rarely if ever compared to alternative models. Here, using data from \citet{walde2022higher} for \emph{Quercus robur}, we fit a non-linear model and compare the common model where cool and warm treatments interact with, such that longer cool temperatures mean more warm temperatures are required for leafout, with a model where longer cool temperatures change the start date of forcing (effectively, were the date of endo-dormancy break is shifted by cool temperatures, not the required warm accumulation). We can also add something about which fits better if I work on that .... This latter model is arguably more in line with the current biological model but rarely fit.} 
\label{fig:multimodelexps}
\end{figure}

\end{document}



