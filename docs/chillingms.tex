\documentclass[11pt]{article}
\usepackage[top=1.00in, bottom=1.0in, left=1in, right=1in]{geometry}
\renewcommand{\baselinestretch}{1.1}
\usepackage{graphicx}
\usepackage{natbib}
\usepackage{amsmath}
\usepackage{parskip}

\def\labelitemi{--}
\parindent=0pt

\begin{document}
\bibliographystyle{/Users/Lizzie/Documents/EndnoteRelated/Bibtex/styles/besjournals}
\renewcommand{\refname}{\CHead{}}

Plant leafout each spring has shifted weeks earlier in many regions due to warming from anthropogenic climate change, with consequences for a suite of ecosystem services, including carbon storage CITES.  What exact processes drive this trend, however, has come under increasing debate as recent research suggests winter warming may slow or stall this advance CITES. Such reports focus on a two-step model of leafout where plants first require cool winter temperatures, often called `chilling,' before they can accumulate enough warm temperatures---`forcing'---to leafout each year. But this model of leafout---especially its chilling component---is only one of many models proposed (at least XX exist, see CITES) since the concept was first introduced. 

Current models of chilling predict a full suite of possible future leafout. Chilling in the same location under different models can easily forecast significantly increased or greatly reduced chilling and thus greatly advanced or slowed leafout (CITES GuyChuine). Such extreme variability of spring phenology predicted from current models suggests fundamental and major gaps in our understanding. Indeed, new research suggests major flaws in these models as currently applied, including multiple papers now suggesting estimates of `chilling' could easily be artifacts of poor models or correlated observational climate data (CITES). 

Now may be an opportune time to address the problems in models of chilling that lead to extremely different forecasts. Because accurate forecasts of spring phenology are critical for carbon storage, many crops and a number of other services important to humans, there is high interest in improving chilling models. At the same time, new results from molecular and cellular studies of dormancy are providing new insights into when and how chilling works, which could rule in---or out---some of the myriad current models. 

Here, we review the concept of chilling, its origins and potential problems, as well as new opportunities for major advances. In particular, we discuss how new insights from molecular and cellular studies combined with modern approaches to building biological models could revolutionize our understanding of chilling. We close by outlining the major implications better models of leafout could have for our fundamental understanding of plant dormancy and for forecasts of climate change. 

\section*{What is chilling?}

How plants in temperature-limited systems avoid leafout during warm spells in the winter has long been debated by plant biologists. Most work to date has focused on the idea that plants enter some form of dormancy in the fall, which is then released before spring warm temperatures begin. This idea hypothesizes that the slow accumulation of cool temperatures---or chilling---over the winter extends through periods of short warm spells in the winter and thus prevents leafout before spring. 

Much of our fundamental understanding of chilling comes from studies on temperate woody plant crops where chilling can be critical to fruit production. Peach trees planted into warmer climates well outside their range (e.g., planted in Florida, Israel) often have extremely low fruitset because most flower buds do not burst (CITES). Initial studies of this phenomenon with related experiments---where cut ends of dormant branches exposed to cooler temperatures in chambers burst more fully and more quickly---underlies most of the models of chilling used today for crops and wild tree species CITES. %, but these have rarely---if ever---has been well tested. 

The term `chilling' is now used across numerous fields to refer to a process where dormant buds exposed to cool temperatures accelerate a phenological event that later occurs after warm temperatures. Focused on how chilling can accelerate events, researchers have calculated `chilling' required for leafout of forest trees from satellite measures of greenup, ground observations, and from similar cutting experiments to those used for peaches CITES.  % A suite of studies on forest trees have also estimated chilling using cuttings of tree branches during the winter in experiments where `chilling' is manipulated either in controlled environment chambers or through the sequential removal of cuttings CITES. While most of these studies focus on leafout, a similar suite of studies on crops has focused on flower burst. 
Alongside these more macro-scale studies of chilling, molecular approaches have also examined chilling, with a wealth of studies on vernalization in \emph{Arabidopsis thaliana}---cool temperatures required for flowering CITES. These studies generally use controlled temperatures to vary the hypothesized amount of chilling---similar to cutting studies of forest trees and woody crops---then examine molecular and cellular responses CITES. Versions of plants with knock-outs of specific genes have helped identify genes that underlie dormancy and other processes related to chilling CITES. 

% Add reference to figure showing several different models here.
Today these studies have led to over XX models of chilling that release plants from dormancy. Though early debates considered whether plants were truly dormant or only growing slow (`rest' versus `quiescent,' CITES), today most research assumes a model with two phases of dormancy. In the first phase---endo-dormancy---plants wait for complete chilling accumulation and cannot respond to external warm periods (thus the `endo' part of the term). Once they have accumulated the appropriate amount of `chilling' they are `eco-dormant,' and will leaf or flower in response to a sufficient thermal sum (often called `growing degree days' in some models). In these models, chilling can only be accumulated under certain temperatures---often above zero but below 10\degree C---with certain temperatures being optimal for the most rapid accumulation of `chill units,' where some unknown sum of sufficient chill units breaks endo-dormancy. This model mirrors growing degree days in many ways, where a lower temperature (e.g., 5 or 10\degree C base temperature) is too cold for growing degree units to accumulate and plants need some total sum of thermal units to leafout or flower, but has added complexity given the importance of both a lower and upper temperature threshold for `chill units,' and the hypothesized diversity of these temperature thresholds and sums across species and populations. Most assume different species require different sums of chill units, may have different lower, upper and optimal chill temperatures, and some studies have found variation across populations---with populations in more mild climates where warm interruptions in the winter are more common requiring more chill units than populations in areas with cold winters that rarely warm much above zero before spring CITES. 

This complexity of chilling has led to varying definitions. In most models the term `chilling' only occurs when plants are in a specific physiological phase and receiving temperatures known to induce chilling. In an experimental context, however, the term `chilling' often refers to a treatment without knowing the actual physiological phase of the plant CITES. For example, cuttings or buds are often chilled at 5\degree C for 6 weeks in a `chilling' treatment, then transferred to warming `forcing' conditions; the actual physiological transition may occur in the cool `chilling' treatment or much later in the warm 'forcing' conditions. While this may seem like a small terminology issue it actually belies one of the major problems with our understanding---and thus modeling---of chilling today CITESettinger. 

\section*{The problem with chilling} 

Experiments and models focused on chilling both describe an unobserved process CITESchuine, for which current data provides limited insights. Some of the early models from peaches, on which many further models seem to rely, used experimental data of percent leaf or flowerburst as evidence that chilling has been met then attempt to identify optimal and possible temperatures from such studies CITES. Taking these estimated temperatures, they then used field observations of percent flowerburst for plants across a wide range of climates---including those well outside the natural range where percent flower burst in many years was low---to estimate the total chilling needed for different cultivars CITES. In this way, separate datasets were used to estimate different parameters of what should be one coherent model, but never attempted to model all the data together. This approach hints at the non-identifiability of current models given our data and knowledge. 

% Current methods use limited data to estimate a hypothesized accumulation that triggers an unobserved event---currently often described as the release of endo-dormancy---which then leads to another accumulation that leads to leafout or flowering. 

% Add reference to figure showing several different models after 'many variants on a more complicated model of chilling'
Current models of chilling generally try to estimate a large number of parameters with very little data. A simple model would need to estimate the minimum and maximum temperatures that allow chilling to accumulate (two parameters) and the total sum of those temperature units needed to trigger endo-dormancy break (one additional parameter for three total). If we assume percent leafout or flowerburst signals sufficient chilling, an extremely large number of experimental treatments would be needed to estimate this (JA help?) and require additional assumptions, such as when chilling accumulation begins. Given experiments and models have suggested many variants on a more complicated model of chilling---for example minimum, maximum and optimal temperatures, or high temperatures that reduce previous accumulation (CITES)---current data are far to little for the models suggested. Further, recent model have often relied on even less data with many current methods using only the timing of leafout (or flowering) to attempt to estimate a model of chilling for different species or locations and project it forward to understand effects of anthropogenic climate change CITES. % Perhaps not surprisingly then, which model is deemed best varies strongly by method and approach CITES, with no clear pattern. 

One way to increase the amount of data used to estimate chilling models would be to include both experimental and observational data together in one model, but this has never been done to our knowledge. Instead, models often infer parameters from other data or studies, then fit only certain parameters of their models to a dataset at hand CITES. While conditions in observational and experimental are often different---for example, many experiments apply cold temperatures in the dark, while photoperiod shifts each day in observational data---continually fitting the resulting data separately slows progress towards a coherent model and contributes to the increasing diversity of proposed models today. 

\section*{New molecular insights could reshape the field and its models} % START HERE ... TRY to finish THIS section if possible without going to the literature (just do it quickly? Then finish 5 and maybe 6; then return to lit and flesh this section out more after other sections done). 

% Cite molecular figure ... 
Molecular insights have long helped crop and forest tree models of chilling CITES, but recent findings appear especially promising to build a more unified and coherent model. Decades of work on the vernalization pathway of  have outlined the pathways---and genes---that lead to flowering only after winter's cool temperatures in biennial populations of \emph{Arabidopsis thaliana}. While previous models of chilling have attempted to integrate some of these insights into tree species CITES, they often resulted in adding additional, more complex, models of the chilling to the long list of existing models. Yet new cellular and molecular results could potentially reduce this complexity, and help narrow the set of possible models. 


...and has worked more on the cellular processes that may control chilling. 

These results also span an increased diversity of species (e.g., herbaceous, woody, perennial bulbs). 

... New molecular insights hold promise to remove part of the non-identifiability, but the field needs to advance to take them on and evaluate them carefully. This is especially important that cellular and molecular insights will likely come from different species than those often modeled and possibly for different events (e.g. for flowering when the field models leafout). 

\section*{Building a better model of chilling} 

Bad things ...
And then there are attempts to estimate chilling using observational data in crops (but often when planted outside range) and forest trees ... and basically always based on assumptions from existing models and/or experiments (peaches again).


\end{document}
