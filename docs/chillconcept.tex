\documentclass[11pt,letter]{article}
\usepackage[top=1.00in, bottom=1.0in, left=1.1in, right=1.1in]{geometry}
\renewcommand{\baselinestretch}{1}
\usepackage{graphicx}
\usepackage{natbib}
\usepackage{amsmath}
\usepackage{gensymb}
\usepackage{hyperref}

\def\labelitemi{--}
\parindent=0pt

\begin{document}
\bibliographystyle{/Users/Lizzie/Documents/EndnoteRelated/Bibtex/styles/besjournals}
\renewcommand{\refname}{\CHead{}}

\title{Chilling paper(s) notes}
\author{Lizzie, Justin Ngo, Fredi, Rob Guy, Jonathan Auerbach (hopefully)} % Add Victor? % July 2025 -- this author list is super old, should update
\date{\today}
\maketitle
\tableofcontents
\section{May 2025 outline for second part} 

Also add somewhere:
\begin{itemize}
\item would there be (or have there been?) experiments testing for budbreak that use genetic mutants like they do in a standard genetic analysis? I don't think I've encountered a single paper where the disciplines feel united experimentally. Would proposing experiments like testing for budbreak \% after increasing chilling using a loss-of-function callose synthase mutant (for example) be one way that ecologists and geneticists could merge and potentially reconcile their respectively understood chilling pathways?
\item so 1) molecular people can maybe do mutant combinations of the genes are hypothesized to have the biggest weight thus far, 2) predictive models like that of Wilczek et al. and Satake et al. are generated to compare how genotypes might respond to different chilling conditions, and 3) these models are applied in phenological experiments to assess their validity?
\item Isabelle was saying she does not use a sequential model and does not model that more chilling means you need less forcing. She models competancy now. We had a long back and forth and what I took away was that her models need competancy to explain some extreme conditions -- like Fagus leafout with almost no chilling but longer photoperiod -- but it was not clear you need competancy to explain most natural conditions. So I wondered whether this sort of differentiation is something we should make in the paper. Maybe we suggest modelers working with natural conditions use sequential or the simplest model possible, but say something else about modeling extreme conditions -- and maybe those folks doing extreme conditions should work on predicting mutants?
\end{itemize}


The path forward should ...
\begin{enumerate}
\item Improve experiments
\begin{enumerate}
\item Redefine chilling for improved experiments
\begin{enumerate}
\item Be careful in definitions of chilling (and vernalization) that define it in ways that assume more than we actually know 
% Redefine chilling as an unknown physiological process---similar to `floragen' period with vernalization, we need to be clear we have no marker % maybe call it `forcagen' or `chillagen' as the thing that allows forcing to start, but we don't know what it is---similar to `floragen'
\item So treatments should not be be referred to as `chilling' and `forcing', just call them cool and warm treatments. 
\end{enumerate}
\item Begin a new generation of experiments that measure molecular stuff (this is critical)
\begin{enumerate}
\item More experiments that bridge to the field (like Wilczek, and Satake)?
\item More experiments that use a range of temperatures 
\end{enumerate}
\item Expect more experiments that do not measure molecular stuff (`old-generation'), measure more stuff
\begin{enumerate}
\item Measure cold hardiness
\item Make all chilling treatments include light or include light/dark as factor
\end{enumerate}
\end{enumerate}
\item Improve (process-based?) models to represent current understanding and lack thereof 
\begin{enumerate}
\item We need to be less overly confident in what we know to make the best use of molecular insights. So ...
\begin{enumerate}
\item Focus on simple models for now
\item Require uncertainty/non-identifiability to be better studied in models 
\end{enumerate}
\item We need better statistical models for empirical data (experiments?), including:
\begin{enumerate}
\item Multiple hypotheses (include alternative start date as hypothesis in experiments that don't measure the start date)
\item Intent-to-treat effect and the use of instrument variables
\end{enumerate}
\item Fit observational and experimental data together (merge process-based and statistical models)
\end{enumerate}
\end{enumerate}



\section{November 2024 outline} % Early November 2024 in Gobabeb

{\bf To do ... } See new file \verb|_dothisChilling.md|\\

{\sc How to overcome the challenge of chilling in spring phenology models} % The future of spring phenology forecasting
\begin{enumerate}
\item Introduction
\begin{enumerate}
\item Why to care ...
\begin{enumerate}
\item Critical to good forecasts (my current common baby) of forests, carbon storage ...
\item Lots of current work suggests it's changing in ways we cannot totally predict and it's even more critical in today's climate than we thought (decsens etc.)
\item But these new studies make a lot of assumptions about chilling 
\item And we have a gazillion models of chilling, suggesting something is amiss
\end{enumerate}
\item Chilling has long been a struggle for plant physiologists
\begin{enumerate}
\item Debates about rest vs. quiescence in 50s-60s (maybe earlier?)
\item Conundrum of it in 60s-70s and peaches
\item Work in 2012 (Guy), 2026 (Chuine) and more have highlighted how little we know if we want to forecast with climate change
\end{enumerate}
\item But this is a special moment for chilling
\begin{enumerate}
\item More important than ever to understand it.
\item And new molecular insights are suggested we know parts of the story that could reshape how we measure and model it
\end{enumerate}
\item Here we ...
\end{enumerate}
\item What is chilling?
\begin{enumerate}
\item Fundamentally, it's the idea that cool temperatures of dormant buds/plants accelerate a phenological event that later occurs after warm temperatures
\item Considered to have evolved to protect plants from leafout during brief warm snaps in middle of otherwise cold winters. 
\item Applied across a lot of fields and study systems ...
\begin{enumerate}
\item Satellite measures of greenup
\item Small scale cutting studies of forest tree species
\item Also, lots of similar work on crops
\item Molecular studies
\item Lots of models of it for forest trees and crops
\item Generally applies to leafing and flowering, with different fields focused on different events (much of molecular is flowers and vernalization, while forest trees and satellite is almost exclusively for leafout)
\end{enumerate}
\item Experimental context for it: things get chilled at 5\degree C (molecular and phenology people)
\item Models of it define it as what is needed for endodormancy break or leafout ...
\begin{enumerate}
\item Most common model is often sequential (explain it, include optimal chill idea)
\item But lots of variation, alternating, more complex versions of alternating (parallel etc.), whatever Luedeling is called. 
\end{enumerate}
\item And then there are attempts to estimate chilling using observational data in crops (but often when planted outside range) and forest trees ... and basically always based on assumptions from existing models and/or experiments (peaches again). 
\end{enumerate}
\item The problem with chilling
\begin{enumerate}
\item Refers to either:
\begin{enumerate}
\item a latent accumulation before an event
\item an experimental treatment
\item But the problem with this dichotomy is rarely if ever acknowledged
\end{enumerate}
\item Non-identifiability of chilling (unrecognized)
\item Experimental and observational data often don't match
\item Resulting models from this research are poor, and thus we build a million different models
\end{enumerate}
\item New molecular insights could reshape the field and its models 
\begin{enumerate}
\item Molecular has always been a focus, but new results are particularly exciting and more studies are coming out across more species and the pathway is coming together (cite figure)
\item We mention two major ones: callose and time-dependent stuff
\begin{enumerate}
\item Review callose
\item If callose is correct, it may rule out long periods when chilling and forcing should act together (e.g., parts of alternating and parallel models) but also times when they may co-act and warm temperatures can negative `chilling' because there is also .. 
\item Time-dependent growth (Zhu)
\end{enumerate}
\item New molecular insights hold promise to remove part of the non-identifiability, but the field needs to advance to take them on and evaluate them carefully given they will likely come from different species and possibly different events ....
\end{enumerate}
\item Be more aware of assumptions and limitations (including non-identifiability) of current models and current data
\begin{enumerate}
\item Model assumptions should be more explicit
\begin{enumerate}
\item How is chilling and endodormancy defined? 
\item What parameters are non-identified and how variable can these be in the model.
\item What assumptions are made about some of the known potential complexities of chilling (see Box) such as photoperiod, induction temperature, species variability etc..
\item Treat current models as bad guesses, and consider building ones with fewer parameters to test against complex ones (explain value of this, and push this idea! Refer to Box to suggest we may miss other factors if we press on with complex non-identified models only)
\end{enumerate}
\item Acknowledge limitations of current data
\begin{enumerate}
\item Remember that chilling treatments in experiments are not the same definition as conceptual concept of chilling (and models)
\item Be careful using any observational data to estimate chill 
\end{enumerate}
\item And find better data
\begin{enumerate}
\item Experiment should be explicit: How is chilling measured? What is photoperiod, induction temperature etc. 
\item Look for observational data outside the range of species with extremely different chill
\end{enumerate}
\end{enumerate}
\item Interdisciplinary efforts to bridge experimental-observational divide (people always say you should be more interdisciplinary, but we really should be here)
\begin{enumerate}
\item Models to date at best use one experiments or observational data to inform the other, but do not build models designed to predict experimental and observational data together; but we could and we should.
\item Building such models would benefit from far greater interdisciplinary efforts, bridging:
\begin{enumerate}
\item Phenology observations and experimental and modeler folks
\item Molecular experimentalists and modelers 
\item Crop biologists and their modelers 
\item Hardiness folks
\end{enumerate}
\item Start building a metanalytic database (see table in Supp perhaps) that can bridge these fields by including all conditions and also what exactly is measured as induction and endodormancy break etc. 
\item Make testable predictions from these advances, and test them 
\end{enumerate}
\end{enumerate}


Box: What we know matters to chilling ... and might matter. 
\begin{enumerate}
\item What are we fairly sure matters (across many studies and species)
\begin{enumerate}
\item Time matters ... is it only time, no also time at different temperatures: $<0$, 0-10, $>10$
\item Temperature ... 
\end{enumerate}
\item What additional factors likely matter, but we're not sure which species or how big effects are...
\begin{enumerate}
\item Stuff related to population/site differences
\begin{enumerate}
\item Fall temperatures (don't forget we have OSPREE papers on induction temperatures)
\item Photoperiod
\item Population variation
\end{enumerate}
\item Stuff related to temperature variability 
\begin{enumerate}
\item Subzero temperatures (Fredi's paper, but see OSPREE lit where old temperature did not matter)
\item Intermittent warm periods
\item This is why hardiness has come up more 
\end{enumerate}
\end{enumerate}
\item Species identity ... it matters! Consider focusing on one model species and race models against each other. 
\end{enumerate}

\clearpage

\section{Next steps: from alt 2022 paper}

{\bf Not applicable to totally new version of paper written above, but saving here in case Fredi or someone does want to do any of this...}\\

{\bf Make tables \& figures?} ... Update old notes below to do that. 

\begin{enumerate}
\item Develop a good diagram and understanding of the callose model... now
\item Keep meeting with Auerbach -- schedule when all three can make it -- hopefully he comes up what we can say, can't say, assumptions that would help
\item Fredi is reading the freeze lit; Lizzie will put name of refs in a new issue (and share OSPREE folder) -- see \emph{When reading papers .... } below
\item Lizzie is reading the old model literature to write up the history and get assumptions (already has most files, need to read and organize notes) -- \emph{When reading papers .... } below
\item Not yet done -- maybe read dormancy induction studies in OSPREE 
\end{enumerate}

\emph{When reading papers .... }
\begin{enumerate}
\item What they think of and how they deal with endo/eco versus chill/force (versus just rest)
\item How they \% versus days?
\item What they could allow us to assume ... what assumptions they support or violate?
\item Which species
\end{enumerate}

\section{Old outline} % Lizzie in Nov 2024; This outline is mostly from December 2023

Current models of spring tree phenology are non-identified and thus semi-imaginary

What do we want to take away from this paper?
\begin{enumerate}
\item Recognize current chilling models are extremely flawed ... based on fruit tree crops, non-identifiable and few studies conclusively show two-stage for many species (we think) 
\item Need to do better experiments! (And build fewer chilling models.)
\end{enumerate}

\begin{enumerate}
\item Introduction
\begin{enumerate}
\item We all think dormancy is critical to many forest trees (and thus climate change), but is dormancy real? (Baby: Good forecasts)
\item Concepts of dormancy: para, endo, eco (Lang ref)
\item Maybe review quickly evidence against in (and maybe for it) -- Cook et al. ref (Fredi's ref)
\item Is it real or just helpful? -- Warewolf!
\item Here we ... briefly review two-stage model  (maybe where came from) and its general application to forest trees), lay out how it's actually non-identified given (all?) current data, review critical evidence/info we have that could build towards an identified model, especially given new hope to measure the mechanism of endodormancy (callose/hormones)
\end{enumerate}
\item Review of two-stage model
\begin{enumerate}
\item Logic of the two-stage model
\begin{enumerate}
\item Dealing with cold climates (fit in continentality here?)
\item What would evolve to avoid frost? (and still compete for resources)
\end{enumerate}
\item History of the two-stage model (and related models) --  they \% more than time to BB
\item Application of two-stage model to forest trees
\begin{enumerate}
\item How it's been applied
\end{enumerate}
\item New evidence for two-stage (callose, hormone) -- work through what the current callose model predicts, and doesn't and also how well it fits to two-stage
\item How good is this model? % Evidence for and against -- or neither -- for forest trees (needs better section header)
\item Well, it's non-identified ...You can't estimate that many parameters
\item Because it's non-identified it's lead to a whole slew of models -- many, many models! But we can't actually identify the parameters. 
\end{enumerate}
\item Critical evidence to build a better model (what assumptions can we make)
\begin{enumerate}
\item Time matters ... is it only time, no also time at different temperatures: $<0$, 0-10, $>10$
\item Temperature ... 
\begin{enumerate}
\item Optimal chill -- evidence for this -- \% versus time to BB: what tells us what? 
\item Subzero temperatues
\item Intermittent warm periods
\item Fall temperatures (don't forget we have OSPREE papers on induction temperatures)
\end{enumerate}
\item And photoperiod 
\begin{enumerate}
\item at induction
\item During ...
\item In releasing endodormancy (check OSPREE papers with non-0 chill photoperiods)
\end{enumerate} 
\item Species diversity... 
\begin{enumerate}
\item Crops vs. wild
\item Species diversity 
\begin{enumerate}
\item Assume universal model with different parameters for each species or population (mention parameterization for individual species), kind of weird, no?
\item Seeds don't have a universal model
\end{enumerate}
\item Subtropical trees -- Should they belong to the same model?
\item where this subtropical evidence fits and how could we start to build a framework to predict different models for different sets of species  % ... and what should happen with callose in subtropical
\item Population diversity (including continentality -- find the Doug fir paper on chilling being higher in coastal areas)
\item How good is the evidence for population differences? Chilling differences at site; induction temperature differences (and what if photoperiod matters?)
\end{enumerate}
\end{enumerate}
\item Future directions ... 
\begin{enumerate}
\item Acknowledge chill and force treatments do not usually measure chill and force
\item So we need better experiments
\begin{enumerate}
\item More work with molecular pathways at the same time you do chilling/forcing experiments
\item Maybe more with \% budburst or otherwise looking for evidence of endodormancy
\end{enumerate}
\item Critical experiments to deal with non-identifiability that we can do now ... if we have figured any of these out
\item Jump on breakthroughs in callose/hormones to improve experimental measurements of endodormancy -- check if it is similar across species
\end{enumerate}
\end{enumerate}


\clearpage

\section{Meeting notes}
\subsection{8 December 2023: Lizzie meets with Loren Riesberg}

Full notes in my gray and green notebook, but some major points to rememeber here:

\begin{itemize}
\item I started by describing general models of chilling and forcing we have and Loren quickly said that it indeed sounded like methylation, and  methylation patterns would be a good way to tell these models apart. 
\item Good work by Caroline Dean (Innes Centre, UK) on \emph{Arabidopsis}. Good recent talk here: \url{https://www.youtube.com/watch?v=N4qnj7cSZdo} ... he suggested to then look at what papers have cited Dean work in Populus and peach. This work is all about silencing that leads to one step (plant shutdown) and silencing that then leads to growth. 
\item He thinks Dean work shows that prolonged and lack of hot spikes promotes FLC
\item Read up, then have a call with Dean, and then a call with a Populus group. 
\begin{itemize}
\item FLC (MADS box) represses (silences) genes that cause flowering: See around minute 12 she shows FLC slowly being turned off during the winter. 
\item VIN3 is the cold-regulated protein
\end{itemize}
\item `Epigenetic clock' that goes forward and backward
\item He expected subzero temperatures do not do anything, but could ask Rob's opinion (IMHO, it makes sense to me that subzero would not matter to dormancy as it doesn't seem to give much extra info or would be needed to do {\bf do} anything: if you build callose during cool temperatures and do nothing below zero that works! Why do anything sub-zero for a process like dormancy?)
\item He expects callose could be built and degraded throughout the winter (see below, goes with high temperature negation model)
\item I asked about evidence that dormancy induction temperatures matter and he replied that he expects some plants start dormancy processes, then stop, maybe go backward, start again ... and that could look like the idea that dormancy induction temperatures matter. 
\item For modeling, we really need to know how fast is regression compared to accumulation -- see if Dean has looked at this. 
\end{itemize}

{\bf My thoughts after this meeting:}
\begin{itemize}
\item Most papers I have looked at seem to be finding orthologs for Dean's work. 
\item Loren expected (and Dean seems to expect/find for populations) differences in what temperatures matter across populations and species
\item Models ... 
\begin{itemize}
\item Molecular results do support idea of accumulation of cool temperatures. 
\item Chilling probably does not start as soon as we think it does (for example, August or September). It starts once there is transcriptional shutdown, and callose built ... though I am not sure anyone has shown how long these processes to take, which would be good to know. 
\item In Dean talk around minute 23 she says that VIN3 is very slow ... she says it takes weeks and weeks to silence. VIN3 takes 6 weeks of cold. So the cold accumulation is slow at the molecular levels. 
\item Models with high temperature negation may make sense. 
\item Parallel model where plants are gaining chilling and forcing at once seems wrong. Since there is callose that needs to be unblocked for budburst to start, and we know there genes turned on for making callose, and turned off to degrade it, this seems a bad model.
\end{itemize}
\end{itemize}


\subsection{4 December 2023: We meet again on this!}

In the morning we had a big conversation about what is the model for leafout, especially for `chilling.'  Questions we ended up with:

\begin{enumerate}
\item Dormancy depth as measured by forcing units needed (days at 20\degree C in Fredi's 2021 paper for example) -- is that an okay response variable or do we not understand forcing enough and days should be the response? If so, how do we model it altogether?
\item Should we model \% or days or \% over days?
\item Also, exciting short conversation on bet-hedging and leafout!
\begin{enumerate}
\item Should we model buds as cohorts and include variability as an expected response? ({\bf Bet-hedging and buds}) Could think of cohort as having varying dormancy depths and thus needing different forcing units
\item In Fredi's experiment, some species never burst much about 50-60\% ... 
\item If bet-hedging is happening but we assume \% budburst is related to dormancy depth, we may confuse the two completely. 
\end{enumerate}
\end{enumerate}

Then we chatted with Jonathan Auerbach -- notes mainly in my green/gray notebook but a few here:

\begin{enumerate}
\item How to model experiments and observations together (Lizzie's eternal dream)
\item Renewal theory is basically the bucket model
\item What could we hope to do?
\begin{enumerate}
\item Figure out what our current experiments DO show. For example, can we show that chilling affects forcing? (Can we show it's not just time, but that temperature matters?)
\item What can we say about chilling and forcing with current data? Given this is probably not much given identifiability issues ...
\item What assumptions can we make that would allow us to say more? 
\item What are the critical experiments that would test assumptions or really advance things
\end{enumerate}
\end{enumerate}

In the afternoon we brainstormed: {\bf Points to cover in paper and/or /disturbing problems/issues:}

\begin{enumerate}
\item Is the model two-stage or parallel (or both, but depends on species ID)?
\item We cannot (and do not) fit the experimental and observational data together.
\item Models are basically made-up based on old studies ...
\item Where does evidence/theory come from for current models of chilling?
\item Models are non-identifiable
\item Hardiness vs. dormancy
\item Temperature fluctuations
\item Influence of time without any other effects
\item Photoperiod -- during `chilling' and during `forcing'
\item List out big things that matter and maybe smaller things that matter
\item Transition time from endo to ecodormancy -- is it instantaneous or gradual? What do we know from callose?
\item What is the callose model? What temperatures is it degraded at? When is it built? What exactly is it blocking between cells (hormone etc.)? -- How does it compare to the two-stage and parallel model?
\item Hormones and dormancy?
\item Molecular evidence etc.
\item Corollaries with seeds
\item Dormancy depth (forcing needed after putting in warm temperatures) since budset through to next budburst
\item GDD model is based on development -- but do we know if this is correct? It may be more like a timer that also runs (differently for each species). What is structural growth? Maybe none of it? GDD is from crops and is mainly structural growth, even though we don't think structural growth is happening during budburst? 
\item Basically, what's chilling? And what's forcing? 
\item What we don't know with chilling
\begin{enumerate}
\item What temperatures accumulate chill? Can it be really low? Can it be really high? What do freezing temperatures interrupting this do?
\item Can accumulated chill be negated?
\item Is it just time?
\item Just to confirm: Chilling happens below 10 C only/mostly?
\end{enumerate}
\item Relevance of chilling in subtropical trees. 
\end{enumerate}


We need to. ...	
\begin{enumerate}
\item Understand the progression of the major old literatures that lead to the Utah model and the other model (Fishman? This is precursor to dynamic chill?). 
\item Confirm how these models were extrapolated to forest trees (or at least compare them to the current forest tree models) -- we could just ask Isabelle about this and check Harrington papers and Murray et al. 1989
\item {\bf Remember} to Never re-read chill models, just read old notes I have
\item Divide up lit review tasks ...
\begin{enumerate}
\item Fredi does all OSPREE papers with negative chill or freeze... (Lizzie sends them to him, see getchill.R)
\item Somebody does dormancy induction OSPREE papers
\item Lizzie does the old modeling papers
\end{enumerate}
\end{enumerate}

Tomorrow! (4 December 2023)
\begin{enumerate}
\item Arrive having read Isabelle's 2016 paper
\item Make a broad outline ... some points already
\begin{enumerate}
\item Callose model vs. the current models (Lizzie has old notes, see notes within \verb|chillingrefs_holidayedition.pdf|
\item What determines dormancy release?
\begin{enumerate}
\item Time alone
\item Time at different temperatures: $<0$, 0-10, $>10$ ... 
\item What do high temperatures do to `chilling'
\end{enumerate}
\item  \% versus time to BB: what tells us what? 
\item What are feasible models?
\begin{enumerate}
\item What would be the best model?
\item What do what know about what plants can measure?
\end{enumerate}
\end{enumerate}
\item Officially divide up tasks and decide when feasible/best to do them
\item If time allows, work on callose model/papers
\end{enumerate}

% Kovaleski paper:
% https://www.biorxiv.org/content/10.1101/2022.09.15.508138v1.full
\subsection{Thinking ahead on 3 December 2023 before meeting tomorrow}

I think a good lit review could be in order for this paper ... but we'd need to think hard about how to do it. We'd likely want to include endo and ecodormancy (how well they measured it and included it ...). Then I just worked up  \verb|chillingrefs_holidayedition.pdf|

\subsection{Notes from very brief chat with F. Baumgarten on 1 November 2022 about writing a concept paper on chilling}

{\bf Next steps:}
\begin{enumerate}
\item Fill in outline, esp. the what we know on chilling
\item Pull the OSPREE papers and compare across the different questions (subzero temps, intermittent warm etc.) -- are there any consistencies?
\item Read all the old papers I have pulled!
\item Read all our old notes, organize ... 
\end{enumerate}

{\bf \Large Outline...} Coming out of the dark ...  the critical role of photoperiod in chilling

Fig 5.1 — Baumgarten discussion 

\begin{enumerate}
\item What is the known (and possibly known) biology of of chilling?
\begin{enumerate}
\item Optimal chill
\item Intermittent warm periods
\item Subzero temperatues
\item Fall temperatures
\end{enumerate}
\item What are the chilling models?
\begin{enumerate}
\item Utah 
\item Chill portions ...
\item Adapting to each species' optimal chill (and Lizzie's worries over this)
\end{enumerate}
\item So what do we really know (esp. maybe to improve chilling models or at least approach them with caution when we interpret them) ... What are the biases/uncertainties given all this? What is definitely known?
\begin{enumerate}
\item If we don't know chilling then we have issues with forcing also ... 
\end{enumerate}
\end{enumerate}

\newpage
\subsection{Notes from whiteboard (December 2023) written up in April 2024}

Things we really need to know to merge chill/force models and molecular/physiology understanding:
\begin{enumerate}
\item  When is callose built and for how long is it active?
\item  Callose: Is it partly or fully blocking plasodesmata?
\item Callose: Is is reversible?
\item So, is endodormancy when callose ...
\begin{enumerate}
\item  is already built and not degraded?
\item  callose is being degraded (which would fit with `chilling')?
\item  in some feedback loop (constantly built)?
\item  built and degraded but enough to block cells?
\end{enumerate}
\item So, when chilling have an impact?
\end{enumerate}

\end{document}