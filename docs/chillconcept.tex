\documentclass[11pt,letter]{article}
\usepackage[top=1.00in, bottom=1.0in, left=1.1in, right=1.1in]{geometry}
\renewcommand{\baselinestretch}{1.1}
\usepackage{graphicx}
\usepackage{natbib}
\usepackage{amsmath}

\def\labelitemi{--}
\parindent=0pt

\begin{document}
\bibliographystyle{/Users/Lizzie/Documents/EndnoteRelated/Bibtex/styles/besjournals}
\renewcommand{\refname}{\CHead{}}

Notes from very brief chat with F. Baumgarten on 1 November 2022 about writing a concept paper on chilling ....\\

{\bf Next steps:}
\begin{enumerate}
\item Fill in outline, esp. the what we know on chilling
\item Pull the OSPREE papers and compare across the different questions (subzero temps, intermittent warm etc.) -- are there any consistencies?
\item Read all the old papers I have pulled!
\item Read all our old notes, organize ... 
\end{enumerate}

{\bf \Large Outline...} Coming out of the dark ...  the critical role of photoperiod in chilling

Fig 5.1 — Baumgarten discussion 

\begin{enumerate}
\item What is the known (and possibly known) biology of of chilling?
\begin{enumerate}
\item Optimal chill
\item Intermittent warm periods
\item Subzero temperatues
\item Fall temperatures
\end{enumerate}
\item What are the chilling models?
\begin{enumerate}
\item Utah 
\item Chill portions ...
\item Adapting to each species' optimal chill (and Lizzie's worries over this)
\end{enumerate}
\item So what do we really know (esp. maybe to improve chilling models or at least approach them with caution when we interpret them) ... What are the biases/uncertainties given all this? What is definitely known?
\begin{enumerate}
\item If we don't know chilling then we have issues with forcing also ... 
\end{enumerate}
\end{enumerate}

\end{document}