\documentclass[11pt,letter]{article}
\usepackage[top=1.00in, bottom=1.0in, left=1.1in, right=1.1in]{geometry}
\renewcommand{\baselinestretch}{1}
\usepackage{graphicx}
\usepackage{natbib}
\usepackage{amsmath}
\usepackage{gensymb}

\def\labelitemi{--}
\parindent=0pt

\begin{document}
\bibliographystyle{/Users/Lizzie/Documents/EndnoteRelated/Bibtex/styles/besjournals}
\renewcommand{\refname}{\CHead{}}

\title{Chilling paper(s) notes}
\author{Lizzie, Fredi, Jonathan Auerbach on some}
\date{\today}
\maketitle
\tableofcontents

\section{Next steps}

\begin{enumerate}
\item Develop a good diagram and understanding of the callose model... now
\item Keep meeting with Auerbach -- schedule when all three can make it -- hopefully he comes up what we can say, can't say, assumptions that would help
\item Fredi is reading the freeze lit; Lizzie will put name of refs in a new issue (and share OSPREE folder) -- see \emph{When reading papers .... } below
\item Lizzie is reading the old model literature to write up the history and get assumptions (already has most files, need to read and organize notes) -- \emph{When reading papers .... } below
\item Not yet done -- maybe read dormancy induction studies in OSPREE 
\item Meetings scheduled! First one is Leap Day -- 29 Feb 2024!
\end{enumerate}

\emph{When reading papers .... }
\begin{enumerate}
\item What they think of and how they deal with endo/eco versus chill/force (versus just rest)
\item How they \% versus days?
\item What they could allow us to assume ... what assumptions they support or violate?
\item Which species
\end{enumerate}


\section{Current outline}

Current models of spring tree phenology are non-identified and thus semi-imaginary

What do we want to take away from this paper?
\begin{enumerate}
\item Recognize current chilling models are extremely flawed ... based on fruit tree crops, non-identifiable and few studies conclusively show two-stage for many species (we think) 
\item Need to do better experiments! (And build fewer chilling models.)
\end{enumerate}

\begin{enumerate}
\item Introduction
\begin{enumerate}
\item We all think dormancy is critical to many forest trees (and thus climate change), but is dormancy real? (Baby: Good forecasts)
\item Concepts of dormancy: para, endo, eco (Lang ref)
\item Maybe review quickly evidence against in (and maybe for it) -- Cook et al. ref (Fredi's ref)
\item Is it real or just helpful? -- Warewolf!
\item Here we ... briefly review two-stage model  (maybe where came from) and its general application to forest trees), lay out how it's actually non-identified given (all?) current data, review critical evidence/info we have that could build towards an identified model, especially given new hope to measure the mechanism of endodormancy (callose/hormones)
\end{enumerate}
\item Review of two-stage model
\begin{enumerate}
\item Logic of the two-stage model
\begin{enumerate}
\item Dealing with cold climates (fit in continentality here?)
\item What would evolve to avoid frost? (and still compete for resources)
\end{enumerate}
\item History of the two-stage model (and related models) --  they \% more than time to BB
\item Application of two-stage model to forest trees
\begin{enumerate}
\item How it's been applied
\end{enumerate}
\item New evidence for two-stage (callose, hormone) -- work through what the current callose model predicts, and doesn't and also how well it fits to two-stage
\item How good is this model? % Evidence for and against -- or neither -- for forest trees (needs better section header)
\item Well, it's non-identified ...You can't estimate that many parameters
\item Because it's non-identified it's lead to a whole slew of models -- many, many models! But we can't actually identify the parameters. 
\end{enumerate}
\item Critical evidence to build a better model (what assumptions can we make)
\begin{enumerate}
\item Time matters ... is it only time, no also time at different temperatures: $<0$, 0-10, $>10$
\item Temperature ... 
\begin{enumerate}
\item Optimal chill -- evidence for this -- \% versus time to BB: what tells us what? 
\item Subzero temperatues
\item Intermittent warm periods
\item Fall temperatures (don't forget we have OSPREE papers on induction temperatures)
\end{enumerate}
\item And photoperiod 
\begin{enumerate}
\item at induction
\item During ...
\item In releasing endodormancy (check OSPREE papers with non-0 chill photoperiods)
\end{enumerate} 
\item Species diversity... 
\begin{enumerate}
\item Crops vs. wild
\item Species diversity 
\begin{enumerate}
\item Assume universal model with different parameters for each species or population (mention parameterization for individual species), kind of weird, no?
\item Seeds don't have a universal model
\end{enumerate}
\item Subtropical trees -- Should they belong to the same model?
\item where this subtropical evidence fits and how could we start to build a framework to predict different models for different sets of species  % ... and what should happen with callose in subtropical
\item Population diversity (including continentality -- find the Doug fir paper on chilling being higher in coastal areas)
\item How good is the evidence for population differences? Chilling differences at site; induction temperature differences (and what if photoperiod matters?)
\end{enumerate}
\end{enumerate}
\item Future directions ... 
\begin{enumerate}
\item Acknowledge chill and force treatments do not usually measure chill and force
\item So we need better experiments
\begin{enumerate}
\item More work with molecular pathways at the same time you do chilling/forcing experiments
\item Maybe more with \% budburst or otherwise looking for evidence of endodormancy
\end{enumerate}
\item Critical experiments to deal with non-identifiability that we can do now ... if we have figured any of these out
\item Jump on breakthroughs in callose/hormones to improve experimental measurements of endodormancy -- check if it is similar across species
\end{enumerate}
\end{enumerate}


\clearpage

\section{Meeting notes}
\subsection{4 December 2023: We meet again on this!}

In the morning we had a big conversation about what is the model for leafout, especially for `chilling.'  Questions we ended up with:

\begin{enumerate}
\item Dormancy depth as measured by forcing units needed (days at 20\degree C in Fredi's 2021 paper for example) -- is that an okay response variable or do we not understand forcing enough and days should be the response? If so, how do we model it altogether?
\item Should we model \% or days or \% over days?
\item Also, exciting short conversation on bet-hedging and leafout!
\begin{enumerate}
\item Should we model buds as cohorts and include variability as an expected response? ({\bf Bet-hedging and buds}) Could think of cohort as having varying dormancy depths and thus needing different forcing units
\item In Fredi's experiment, some species never burst much about 50-60\% ... 
\item If bet-hedging is happening but we assume \% budburst is related to dormancy depth, we may confuse the two completely. 
\end{enumerate}
\end{enumerate}

Then we chatted with Jonathan Auerbach -- notes mainly in my green/gray notebook but a few here:

\begin{enumerate}
\item How to model experiments and observations together (Lizzie's eternal dream)
\item Renewal theory is basically the bucket model
\item What could we hope to do?
\begin{enumerate}
\item Figure out what our current experiments DO show. For example, can we show that chilling affects forcing? (Can we show it's not just time, but that temperature matters?)
\item What can we say about chilling and forcing with current data? Given this is probably not much given identifiability issues ...
\item What assumptions can we make that would allow us to say more? 
\item What are the critical experiments that would test assumptions or really advance things
\end{enumerate}
\end{enumerate}

In the afternoon we brainstormed: {\bf Points to cover in paper and/or /disturbing problems/issues:}

\begin{enumerate}
\item Is the model two-stage or parallel (or both, but depends on species ID)?
\item We cannot (and do not) fit the experimental and observational data together.
\item Models are basically made-up based on old studies ...
\item Where does evidence/theory come from for current models of chilling?
\item Models are non-identifiable
\item Hardiness vs. dormancy
\item Temperature fluctuations
\item Influence of time without any other effects
\item Photoperiod -- during `chilling' and during `forcing'
\item List out big things that matter and maybe smaller things that matter
\item Transition time from endo to ecodormancy -- is it instantaneous or gradual? What do we know from callose?
\item What is the callose model? What temperatures is it degraded at? When is it built? What exactly is it blocking between cells (hormone etc.)? -- How does it compare to the two-stage and parallel model?
\item Hormones and dormancy?
\item Molecular evidence etc.
\item Corollaries with seeds
\item Dormancy depth (forcing needed after putting in warm temperatures) since budset through to next budburst
\item GDD model is based on development -- but do we know if this is correct? It may be more like a timer that also runs (differently for each species). What is structural growth? Maybe none of it? GDD is from crops and is mainly structural growth, even though we don't think structural growth is happening during budburst? 
\item Basically, what's chilling? And what's forcing? 
\item What we don't know with chilling
\begin{enumerate}
\item What temperatures accumulate chill? Can it be really low? Can it be really high? What do freezing temperatures interrupting this do?
\item Can accumulated chill be negated?
\item Is it just time?
\item Just to confirm: Chilling happens below 10 C only/mostly?
\end{enumerate}
\item Relevance of chilling in subtropical trees. 
\end{enumerate}


We need to. ...	
\begin{enumerate}
\item Understand the progression of the major old literatures that lead to the Utah model and the other model (Fishman? This is precursor to dynamic chill?). 
\item Confirm how these models were extrapolated to forest trees (or at least compare them to the current forest tree models) -- we could just ask Isabelle about this and check Harrington papers and Murray et al. 1989
\item {\bf Remember} to Never re-read chill models, just read old notes I have
\item Divide up lit review tasks ...
\begin{enumerate}
\item Fredi does all OSPREE papers with negative chill or freeze... (Lizzie sends them to him, see getchill.R)
\item Somebody does dormancy induction OSPREE papers
\item Lizzie does the old modeling papers
\end{enumerate}
\end{enumerate}

Tomorrow! (4 December 2023)
\begin{enumerate}
\item Arrive having read Isabelle's 2016 paper
\item Make a broad outline ... some points already
\begin{enumerate}
\item Callose model vs. the current models (Lizzie has old notes, see notes within \verb|chillingrefs_holidayedition.pdf|
\item What determines dormancy release?
\begin{enumerate}
\item Time alone
\item Time at different temperatures: $<0$, 0-10, $>10$ ... 
\item What do high temperatures do to `chilling'
\end{enumerate}
\item  \% versus time to BB: what tells us what? 
\item What are feasible models?
\begin{enumerate}
\item What would be the best model?
\item What do what know about what plants can measure?
\end{enumerate}
\end{enumerate}
\item Officially divide up tasks and decide when feasible/best to do them
\item If time allows, work on callose model/papers
\end{enumerate}

% Kovaleski paper:
% https://www.biorxiv.org/content/10.1101/2022.09.15.508138v1.full
\subsection{Thinking ahead on 3 December 2023 before meeting tomorrow}

I think a good lit review could be in order for this paper ... but we'd need to think hard about how to do it. We'd likely want to include endo and ecodormancy (how well they measured it and included it ...). Then I just worked up  \verb|chillingrefs_holidayedition.pdf|

\subsection{Notes from very brief chat with F. Baumgarten on 1 November 2022 about writing a concept paper on chilling}

{\bf Next steps:}
\begin{enumerate}
\item Fill in outline, esp. the what we know on chilling
\item Pull the OSPREE papers and compare across the different questions (subzero temps, intermittent warm etc.) -- are there any consistencies?
\item Read all the old papers I have pulled!
\item Read all our old notes, organize ... 
\end{enumerate}

{\bf \Large Outline...} Coming out of the dark ...  the critical role of photoperiod in chilling

Fig 5.1 — Baumgarten discussion 

\begin{enumerate}
\item What is the known (and possibly known) biology of of chilling?
\begin{enumerate}
\item Optimal chill
\item Intermittent warm periods
\item Subzero temperatues
\item Fall temperatures
\end{enumerate}
\item What are the chilling models?
\begin{enumerate}
\item Utah 
\item Chill portions ...
\item Adapting to each species' optimal chill (and Lizzie's worries over this)
\end{enumerate}
\item So what do we really know (esp. maybe to improve chilling models or at least approach them with caution when we interpret them) ... What are the biases/uncertainties given all this? What is definitely known?
\begin{enumerate}
\item If we don't know chilling then we have issues with forcing also ... 
\end{enumerate}
\end{enumerate}

\end{document}