% 30 November 2025
% Now cutting: Take 2 on the paths forward ...

... will require developing models that yield clear and falsifiable predictions. Yet progress may advance slowly given current practices in many models and experiments for estimating chilling, including the increasing the number and complexity of process-based models while fitting overly simplistic statistical models to empirical data. Below we show how these practices have limited developing  models that make falsifiable predictions and outline a pathway forward. 

\subsection*{Develop benchmarks to help discard models} 
A first step to improve the falsifiability of various models is to reduce the number of models to consider. Currently, crop biologists, phenological process-based modelers of forest trees, molecular biologists and hardiness modelers all develop unique and rarely compared models of dormancy and budburst \citep[but see][]{kovaleskipreprint}, highlighting a major problem, but also a major opportunity. Synthesizing models---and their underlying biological understanding of chilling---across the many research fields developing chilling models today would help identify models that are equivalent.  Uniting these models, first by fully defining their assumptions and conditions (e.g., what species are they designed for, what phenological or dormancy phase do they start at?), then comparing their predictions and pushing them to make different testable predictions would help build a unified model of chilling, with implications for better models of budburst, and cascading improvements in forecasts for crop yield and forest carbon sequestration. 
% Current advances in the molecular biology of chilling hint that some components of current chilling models may wrong---based on our current understanding---but fully debating this would take more interdisciplinary efforts. 

Changing how we evaluate and compare models would also aid discarding models. To date, models are compared using different model comparison statistics and different datasets in each paper. This makes it extremely difficult to prioritize models for study and experiments, since one model may be preferred on some data for certain model statistics and another model given other data and statistics. Building standard benchmark datasets that are always tested alongside the same test statistics would alleviate some of this problem, and make it easier to identify why different datasets find different answers, thus aiding model development. Yet discarding models fully likely requires moving away from relying only on model comparisons statistics and towards models that make falsifiable predictions, which will require first embracing how poorly most models used today perform. 
% These differences may explain why some of the simplest process-based models perform best when fit only to observational data, as `natural' field conditions may often satisfy chilling requirements for wild species, allowing models without chilling to excel even though they may fail spectacularly if tested on experimental data. 

\subsection*{Highlight the unknown} 
Perhaps the largest problem with current process-based models of chilling is that they cannot uniquely estimate the most important aspects of chilling. Even in some of the simplest models, estimates of what temperatures accumulate `chilling' and how much chilling is required to shift physiological stages often occupy multiple divergent options. A common example of this is one outcome where the range of temperatures that allow chilling to accumulate is wide and thus the threshold amount of chilling large, or an outcome where the range is smaller and, thus, the threshold lower. Yet researchers often only present one of these outcomes \citep{chuine2016}, versus showing the full range of possible values---that is, the full uncertainty. Highlighting this uncertainty instead of hiding it could quickly help compare models and guide experiments. Thus we suggest research always include routine estimates of uncertainty and clearly state any modeling choices that may have limited insights into the full uncertainty (e.g., limiting the parameter space the model can search when estimating parameter values). Given that this uncertainty only grows with complexity in current models, focusing more on simple models in the near-term may aid progress.
% Adding complexity yields models that fit better generally \citep{statrethink}, but---by making so many possible outcomes possible---this complexity directly decreases falsifiability. 
% Current models of chilling fit better when adding complexity in various ways (see Box: Why has progress on modeling stalled for decades?), but we are no closer to knowing which of these additions is most biologically relevant. 

%Fitting simple statistical models to empirical data that fail to represent the known complexity of chilling also limits falsifiability, by suggesting only one model is possible instead of comparing across alternative hypotheses (Fig. \ref{fig:multimodelexps}). 

{\sc Box: Why has progress on modeling stalled for decades?} % Latent non-identifiability.

% Most major models of chilling were developed over 40 years ago \citep{richardson1974,chuine2016} following several decades of new work---especially on peaches and other temperate tree fruit crops---and much debate \citep{dormtreeproc}. Since then many more models have been proposed \citep{luedeling2012chilling,chuine2016}, but models from the 1970s still often fit very well \citep{basler2016evaluating,chuine2016} and are widely used in major studies \citep[e.g.,][]{richardson1974,chuine2016,ospreebbms}. While we outline several reasons for this, including a disconnect between experimental, observational data and modeling approaches (e.g., process-based versus statistical), one of the main reasons for this stalled progress could be that most models---from the old to new ones---cannot actually estimate the parameters in them. That is, the models are fundamentally non-identifiable.



Researchers tacit approach to these major issues likely has contributed to the expansion of models over the last few decades without any clear advances. While various models have added complexity via the shape of the optimal temperature range for chilling, allowing accumulated chilling to be reduced, shifting the start date of chilling, and/or allowing chilling and forcing to act at once \citep{lued2009,gusewell2017,hanninen1990modelling,Kramer1994}, none of these have swept through the field. These new models of chilling have all added parameters, but none of the parameters added to chilling models in 40 years that have been successful enough to be added to all models of chilling. 

Being clear about model uncertainty, the full number of parameters and how well they fit would advance progress on chilling through multiple routes. First, it would help all researchers in the field recognize what is fundamentally unknown and thus focus more research in these areas. Second, it would highlight which parameters are most often fixed (effectively assumptions in the model) % vvdmSept23: this also relates to falsifiability and 'theory-laden' models
versus fit to data, and to which type of data. With this, more research could easily compare across models and datasets to give better overviews of what is known, assumed, or most often studied (i.e., what parameter model studies try to fit). Given the extensive list of proposed complexities to chilling models \citep{lamb1948effect,campbell1979,leinonen1996dependence,cook2005freezing,Man:2010aa,Jones:2012,Sonsteby:2014aa}, having simpler models (fewer parameters) that are routinely used to compare to more complex models, would likely help the field advance. % Currently, very simple models often outcompete more complex models, suggesting how we advance from these simpler models is also a problem \citep[e.g.,][]{basler2016evaluating}. 
% We argue that better models of chilling are possible given new insights alongside new approaches to how we model---and communicate---models of chilling. Models assumptions and identifiability need to be more clearly explained. This includes clearly defining chilling and endo-dormancy or similar transitions and how they are measured from data used to inform or parameterize models. As most current models of chilling appear non-identified---where all parameters cannot be clearly estimated from given data---more open review and discussion of this could greatly help compare models. 
Highlighting uncertainty in findings from experiments would also aid modeling studies to be more upfront about assumptions and limitations (Fig. \ref{fig:multimodelexps}).

% Highlighting uncertainty in findings from experiments would also aid modeling studies to be more upfront about assumptions and limitations. While experiments often assign treatments as `chilling' and `forcing,' most researchers know that the actual physiological transition may occur in the cool `chilling' treatment or much later in the warm `forcing' conditions, but fail to acknowledge it in their papers or statistical models \citep{ospreebbms,ospreenph2023}. While some experiments rely on a heuristic where budburst that is rapid and includes a high percentage of buds bursting indicates the physiological state associated with forcing (endo-dormancy) has started, the point when this threshold is crossed is rarely clear, and data showing it are often buried in supplements. These practices of hiding uncertainty in results and experimental designs that are effectively non-identifiable for the biological process under study likely contributes to confusion for researchers in the field and drives models with hidden assumptions. A coherent model requires researchers focused on experiments and modelers alike to embrace the current limited level of understanding in the field. % vvdmSept23: and data limitation?





%
%
% Stuff that was previously cut ... 

While this may seem like a small terminology issue it actually belies one of the major problems with our understanding---and thus modeling---of chilling today: measuring and estimating an unobserved process with so many unknowns is extremely challenging \citep[][see also: Box: Why has progress on modeling stalled for decades?]{ospreebbms}. 
% In addition to leveraging molecular insights to determine what model structures best align with our current understanding of physiological dormancy and its release, more efforts to measure compounds related to dormancy release could reshape how we model it. 
%  Testing some of these methods together with cellular and molecular indicators should be a major priority for the field, though researchers will need to evaluate their effectiveness carefully, especially as cellular and molecular insights will likely come from different species than those often modeled and possibly for different events (e.g. for flowering when the many models focus on leafout). % And see BOX on what is wrong with current methods?
% While a number of recent paper papers, have reviewed molecular advances in research related to chilling CITES, sometimes explicitly with an aim to bridge to models of tree chilling (Zhang et al 2023), they often review a diversity of advances. 

% Progress through new experiments could come from experiments that bridge across molecular and phenological methods in other aspects of their design, beyond simply what they measure. This includes more temperature levels for controlled studies to disentangle effects of temperature (including identifying which temperatures are optimal) and time, especially in molecular studies where cold treatments often vary only in duration, not temperature \citep[e.g.,][]{rinne2011,pan2023epigenetic}. Efforts to test molecular results in field conditions (or controlled conditions more similar to natural variation) may provide especially valuable insights   \citep[e.g.][and see below, `Model experimental and observational data together']{Wilczek:2009oa,Burghardt2015}. 

% ...

% ADD ME? Bridging statistical and process-based approaches could make us realized that a simple process-based model might have an equivalent statistical model that can be fit in a more robust framework. 

% These models often assume when to start and stop accumulating, based on much earlier work often done only on crops. Early models of chilling---done mostly on peaches and other woody crops---tried to estimate the physiological phase of chilling by using experimental data of percent leaf or flowerburst as evidence that chilling has been met, then attempted to identify the range of temperatures where chilling accumulates \citep{erez1971}. Taking these estimated temperatures, they then used field observations of percent flowerburst for plants across a wide range of climates---including those well outside the natural range where percent flower burst in many years was low---to estimate the total chilling needed for different cultivars \citep{richardson1974}. This approach of using separate datasets to estimate different parameters of what should be one coherent model laid the groundwork for models today where a number of assumptions or parameters are made directly without be questioned or examined. ... In this way, separate datasets were used to estimate different parameters of what should be one coherent model, an approach that has persisted today, and one that hints at the non-identifiability of current models given our data and knowledge. 


Improved statistical models could challenge some of these assumptions by inching closer to the biology. Simple log transformations better match the non-linear accumulation model of chilling and forcing \citep{decsens}. Models could also relax the assumption that ‘chilling’ and ‘forcing’
treatments correspond to different physiological states by testing whether the start dates may vary with the treatments, as the actual endo-dormancy break could equally occur in either cool or warm treatments. 
Both these alternatives are simple to implement and thus could be included as alternative models that test alternative hypotheses (Fig. \ref{fig:multimodelexps}). 
Integrating other statistical approaches could also relax additional assumptions to provide insights into how chilling works biologically. For example, instrument variable approaches could help in studies attempting to manipulate chilling and forcing where the underlying state is not known \citep{regotherstories}. 
 All these models, however, are likely to be limited in the inference they provide without substantial increases in data \citep{hanninen2019experiments}. 
% While previous models of chilling have attempted to integrate some of these insights into other tree species \citep{landsberg1974apple,Kramer1994}, they often resulted in adding additional, more complex, models of the chilling to the long list of existing models \citep{hanninen1990modelling}. 


% ... Current experimental designs are unlikely to radically challenge models of chilling, but small tweaks may still offer important insights. In particular, experiments considering multiple cool and warm temperature treatments could measure hardiness to improve the model of how hardiness and dormancy interact \citep{kovaleskipreprint}. Cool treatments may also be useful if they tested for the effect of light regimes given the prevalence of cool temperature (`chilling') treatments in the dark to date \citep{ospreebbms}. 



\begin{figure}[h!]
\includegraphics[width=0.9\textwidth]{..//figures/biochemicalModelSketch.png}
\caption{Draft visualization of molecular pathways identified in vernalization and chilling; if we want to keep this, we can update to clarify which have been found in woody species perhaps and match the text a little more. ADD to caption: In the first phase---endo-dormancy---plants are accumulating `chilling' and cannot respond to external warm periods  \citep[thus the `endo' part of the term,][]{chuine2016,lundell2020}. Once they have accumulated the appropriate amount of `chilling' they are `eco-dormant,' and will leaf or flower in response to a sufficient thermal sum (often called forcing, or `growing degree days' in some models). } 
\label{fig:molecular}
\end{figure}

{\sc Box: What we know matters to chilling, and what might matter} % and why we keep going around in circles on negative temperatures?

Studies on how cool winter temperatures affect budburst have identified numerous possible factors that appear to moderate `chilling.' Certainly time and cool temperatures affect budburst timing. Teasing out the effect of time versus temperature has proven difficult, but plants and cuttings (clipped ends of tree branches with one or more bud) exposed for the same amount of time to different constant cool temperatures budburst at different rates and percentages \citep{weinberger,ospreenph2023}. Which temperatures are most effective at providing chilling---estimated as those that lead to either (or both) the highest percent or fastest budburst after exposure to warm (often 20\degree C) temperatures---is a common question addressed in experiments, with different experiments providing different answers \citep{vitasselev,baum2021}. 

This has led to debate recently on whether sub-zero temperatures can provide chilling as one recent study appeared to show \citep{baum2021}, supporting previous research \citep{Jones:2012,Sonsteby:2014aa}, but in contrast to other studies \citep{lamb1948effect,cook2005freezing,Man:2010aa}. These studies, however, vary in a number of additional factors including the species and populations they study, the full conditions under which cool temperatures were applied etc.. This debate highlights the complexity of identifying the dominant controllers on chilling when so many factors vary across studies, and so many factors appear to moderate effects. In addition to the temperature range, temperature variability also appears to play a role CITES. The role of sub-zero temperatures and temperature variability also matter strongly to plant hardiness---what level of cold temperatures plants can withstand without damage, which is induced alongside dormancy in the fall for most temperature species. Recent work by \citet{kovaleskipreprint} has reignited an old debate about whether hardiness influences dormancy CITES---or vice versa. 

% Maybe cut the below? Seems annoying nanny-like text (says the person who wrote it)
These debates---about the importance of sub-zero temperatures and plant hardiness to dormancy and chilling---are not new, and also highlight gaps in the approach many biologists have taken to studying chilling in an era where how well we understand it has major implications due to anthropogenic climate change. Some recent papers  ask similar questions to those of decades ago \citep[e.g.,][]{lamb1948effect,Man:2010aa,baum2021}. While this highlights that such debates may not have been fully settled it can also slow progress. Because of this, we suggest a major need is to build and share databases of all relevant studies---across molecular, crop, and forest biology. Such databases should include information on factors that appear to affect chilling, including dormancy induction temperature, species, provenance location  \citep[as different populations may need different chilling][]{campbell1979,leinonen1996dependence}, thermo- and photo-periodicity of all applied treatments, and any hardiness information.  % Maybe ...  We provide a start to this in the supplement?

% While better documenting the diversity of species and treatments applied in chilling-related studies to date will help, progress may require a greater focus on solving the problem for one species ..... 


\vspace{5ex}
{\sc Possible other Box: The problem with how we measure endo-dormancy release currently} % could also include that we don't measure it much

Traditionally, experiments to test for effective chilling temperature ranges in woody species have used a heuristic where budburst that is rapid and includes a high percentage of buds bursting indicates endo-dormancy release has happened. This method has the benefit of being a useful bioassay---testing if plants can budburst as a metric of an endo-dormancy release (indeed, this method and the term were developed together, CITES)---but has disadvantages in that the method, which works well for some stone fruits, often appears much messier in other species. Because the method relies on fitting a hypothesized curve (or simple threshold) to set a date or level of for endo-dormancy break, how well it performs for many species is often not clear CITES, and thus whether it is an accurate bioassay is similarly unclear. For this reason, others have argued for other methods, such as weighing flower buds \citep{chuine2016}CITES or tracing water reactivation into cells \citep{faust1991bound,Kalcsits2009}. 

% Bad things without a home ...
% And then there are attempts to estimate chilling using observational data in crops (but often when planted outside range) and forest trees ... and basically always based on assumptions from existing models and/or experiments (peaches again).


